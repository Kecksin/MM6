\subsection{Fragen und Antworten zum Text von \textcite{deaux_nation_2006}}
\subsubsection{Welche drei Ebenen beinhaltet Pettigrews (1997) Modell und was ist auf den Ebenen jeweil f"ur Migrantinnen relevant?}
Die drei Ebenen sind \emph{Makro-}, \emph{Meso-} und \emph{Mikroebene} (siehe Fig. \ref{fig:deaux1}).

\begin{figure}[hb!]
        \begin{center}
                \begin{tikzpicture}[every node/.style={ellipse,draw,text centered,minimum size=3cm,text width=6cm}]
                        \node at (0,0) (makro) { {\large \scshape Makro: soziale Strukturen} {\footnotesize Immigrationspolitik, Demographische Muster, soziale Repr"asentationen}};
                        \node [below=of makro] (meso) { {\large \scshape Meso: soziale Interaktion} {\footnotesize Intergruppenkontakt, Stereotype, Netzwerke}};
                        \node [below=of meso] (mikro) { {\large \scshape Mikro: Individuen} {\footnotesize Einstellungen, Werte, Erwartungen, Motivation, Erinnerungen}};
                \end{tikzpicture}
        \end{center}
        \caption{Drei Ebenen nach Pettigrew (1997) aus \textcite{deaux_nation_2006}.}
        \label{fig:deaux1}
\end{figure}

\subsubsection{Was zeigten Esses et al. (2004, 2006) in ihren Studien mit potentiellen Arbeitgebern?}
Migranten werden nicht genauso behandelt wie einheimische B"urger. Bei gleicher Qualifikation werden Migranten schlechter beurteilt. Ein moderierender Faktor f"ur den Zusammenhang sind Einstellungen gegen"uber Immigration.

\subsubsection{Beschreiben Sie den Versuchsaufbau und die Ergebnisse der Studie zu \emph{stereotype threat} \parencite{deaux_nation_2006}. Wie lassen sich die Ergebnisse und deren Bedeutung auf den verschiedenen Stufen von Pettigrews Modell verstehen?}
Karibische Einwanderer der $1.$ und $2.$ Generation wurden in ihrer Testleistung verglichen. Der Test war vermeintlich indikativ (oder \emph{diagnostisch}) f"ur F"ahigkeiten von Afro-Amerikanern oder nicht. Es zeigte sich der vermutete \emph{Interaktionseffekt} von diagnostischem Test und Zeit in den USA: Einwanderer der $2.$ Generation schnitten schlechter ab, wenn der Test vermeintlich diagnostisch war.

Eine "ahnliche Studie wurde mit ethnischer Identifikation durchgef"uhrt. Auch hier gab es einen Interaktionseffekt von diagnostischen Test und ethnischer Identifikation. Bei hoher Identifikation und diagnostischem Test waren die Leistungen schlechter.

In Bezug auf Pettigrews Modell: Meso- und Mikroebene sind notwendig, um diese Ergebnisse zu erkl"aren. Auf der Mesoebene werden die Einstellungen (der anderen gegen"uber der eigenen Gruppe) wahrgenommen. Auf der Mikroebene werden die Leistungen schlechter.

\subsubsection{Wie kann zwischen Mikro- und Mesoebene differenziert werden, und welche Interaktionen sind m"oglich? Beschreiben Sie die Untersuchungen und Ergebnisse zum Selbstwertgef"uhl bei verschiedenen Migrantengruppen (Crocker et al., 1994; Wiley, Perkins \& Deaux, 2006)!}
Es gibt Unterschiede zwischen privater (wie ich meine Gruppe sehe) und "offentlicher Meinung "uber die Eigengruppe (was ich denke, dass andere "uber meine Gruppe sagen). Bei Wei"sen und Asiaten sind diese Beiden in Einklang. Bei Schwarzen nicht.

"Ahnliche Ergebisse gibt es f"ur karibische einwanderer der $1.$ und $2.$ Generation.
