\section{Umgang mit sozialer Diversit"at}
\subsection{Fragen und Antworten zum Text von \textcite{cameron_application_2010}}
\subsubsection{Welche Befunde zeigen, dass sich Diversit"at positiv auswirkt?}
Es gibt positivere Intergruppen-Einstellungen beim Besuch einer ethnisch diversen Schule:
\begin{itemize}
        \item weniger Ingroup Bias
        \item mehr Outgroup Peers als Freunde
        \item positivere Interpretation von Outgroup Mitglieder Verhalten
\end{itemize}

\subsubsection{Welche zwei Annahmen sind grundlegend f"ur multikulturelle Bildungsprogramme?}
\begin{itemize}
        \item Vorurteile gr"unden sich auf Unwissen
        \item Kinder orientieren sich an Normen
\end{itemize}

\subsubsection{Verschiedene Evaluationen von Bildungsprogrammen kamen zu unterschiedlichen Ergebnissen. Was wurde in den unterschiedlichen Studien gefunden und zu welchem Schluss kommen Stephan, Renfro \& Stephan (2004) in ihrer Metaanalyse?}
Manche Studien zeigen einen positiven Effekt, andere Studien keinen Effekt, und manchmal zeigte sich sogar ein negativer Effekt f"ur multikulturelle Bildungsprogramme. Die Metanalyse ergab, dass multikulturelle Bildungsprogramme zwar einen positiven Effekt haben, die Effektst"arken aber nicht so gro"s sind, wie urspr"unglich erwartet.

\subsubsection{Weswegen k"onnen multikulturelle Bildungsprogramme dazu f"uhren, dass sich Stereotype "uber eine andere Gruppe bei teilnehmenden Kindern verst"arken?}
Kognitive F"ahigkeiten beeinflussen die Wirkung von multikulturellen Bildungsf"ahigkeiten und m"ussen deshalb ber"ucksichtigt werden (was aber oft nicht der Fall ist). Bei geringerer kognitiver F"ahigkeit kann es sein, dass Informationen, die nicht zu bereits bestehenden Schemata passen dazu f"uhren, dass diese inkongruenten Informationen nicht erinnert oder verzerrt werden, oder dass nur solche Informationen erinnert werden, die das bereits bestehende Schema st"utzen. Auf diese Weise wird der Unterschied zwischen In- und Outgroup eventuell verst"arkt.

\subsubsection{Was sind die Befunde zu Anti-Rassismustrainings?}
Auch die sind gemischt, und auch die k"onnen sogar zu einer Verst"arkung von Rassismus f"uhren. Au"serdem wirken solche Trainings eher bei der Majorit"at.

\subsubsection{Welche Ziele sollen Diversity Trainings erreichen? Welche Evidenz spricht f"ur oder gegen ihre Wirksamkeit?}
"Ubergeordnetes Ziel ist der Schutz einer Organisation und deren Mitarbeiter vor Verletzung der B"urgerrechte. Erreicht werden soll das durch die Vermittlung von kulturellem Bewusstsein, Wissen und F"ahigkeiten.

Studien mit Studenten, die unter anderen Diskussionen und Rollenspiele beinhalteten fanden einen positiven Effekt f"ur Diversity Trainings. Studien, in denen die (negativen) Gef"uhle gegen"uber "alteren Menschen unterdr"uckt werden sollten resultierten dagegen in negativere Intergruppeneinstellungen.

\subsubsection{Wie wirkt sich zweisprachiger Schulunterricht aus?}
Positivere Einstellungen gegen"uber der anderen Gruppe. Au"serdem werden mehr Gemeinsamkeiten wahrgenommen.

\subsubsection{Was sind die St"arken und Schw"achen der in der Praxis entwickelten Interventionen zur Vorurteilsreduzierung?}
Die drei St"arken sind:
\begin{itemize}
        \item k"onnen in diversen und nicht-diversen Settings eingesetzt werden
        \item sind einfach und benutzerfreundlich
        \item erlauben Anpassung an die konkrete Situation
\end{itemize}

\noindent Der Hauptnachteil ist, dass es keine systematische Evaluation gibt.
