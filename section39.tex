\subsection{Fragen und Antworten zum Text von \textcite{salas_science_2012}}
\subsubsection{Wie wird der Begriff \emph{Training} defininert? Mit welchen Zielen werden Trainings durchgef"uhrt?}

Training besteht aus geplanten und systematischen Aktivit"aten zur F"orderung des Erwerbs von Wissen (\emph{need to know}), Fertigkeiten (\emph{need to do}) und Einstellungen (\emph{need to feel}). Das Ziel von Trainingsma"snahmen sind nachhaltige Ver"anderungen in Verhalten und Kognitionen.

\subsubsection{Welches allegemeine Fazit wird im Hinblick auf die Wirksamkeit von Training gezogen?}
Das allgemeine Fazit besteht aus zwei Teilen. Ersten, \emph{Training works}. Zweitens kommt es stark darauf an, wie ein Training designt, durchgef"uhrt und implementiert wird.

\subsubsection{Welche Punkte sollten vor Beginn eines Trainings beachtet werden? Was wird unter \emph{Training Needs Analysis} verstanden?}
Vor Beginn sollte eine \emph{Training Needs Analysis (TNA)} durchgef"uhrt werden. Warum? Um herauszufinden, \emph{wer} trainiert werden soll, \emph{worin} und in\emph{welchem} organisationalen System. Die TNA besteht aus drei Elementen:
\begin{itemize}
        \item Job-/Aufgabenanalyse
                \begin{itemize}
                        \item KSAs
                        \item \emph{need to know} vs. \emph{need to access}
                        \item Cognitive Task Analysis
                        \item Teamwork
                \end{itemize}
        \item Organisationsanalyse
                \begin{itemize}
                        \item strategische Ausrichtung
                        \item Bereitschaft
                \end{itemize}
        \item Personenanalyse
                \begin{itemize}
                        \item Pers"onlichkeitsmerkmale: Welchen Einfluss haben sie?
                        \item Welche Eigenschaften/demographische Charakteristika haben zuk"unftige Trainees?
                \end{itemize}
\end{itemize}

\subsubsection{Was sollte bei der Ank"undigung eines Trainings beachtet werden?}
Es sollten die M"oglichkeiten, die durch das Training erwachsen, betont werden. Es sollte nicht wie ein Test wirken.

\subsubsection{Welche Rolle spielen Selbstwirksamkeit, Zielorientierung und Motivation der Personen, die an einem Training teilnehmen?}
\begin{itemize}
        \item \textbf{Selbstwirksamkeit} f"uhrt zu besserem Lernen. Training sollte so gestaltet sein, dass Selbstwirksamkeit gef"ordert und best"arkt wird. 
        \item \textbf{Zielorientierung} kann entweder auf \emph{Lernen} oder auf \emph{Leistung} bezogen sein. Lernzielorientierung ist besser und sollte gef"ordert werden. Lernzieloerientierte Trainees sollten Freiheiten und Eigenverantwortung bekommen. F"ur eher Leistungsorientierte Trainees sollte gleichzeitig die M"oglichkeit zu gr"o"serer Strukturierung gegeben sein.
        \item \textbf{Motivation zum Lernen} ist wichtig (wer h"atte es gedacht) und sollte vor, w"ahrend und nach dem Training gef"ordert werden. Es ist hilfreich, dazu die Verbindung zwischen Training und Joballtag herauszustellen. 
\end{itemize}

\subsubsection{In welchen f"ur die Community Psychologie relevanten Bereichen k"onnte die Methode des \emph{Behavioural Role Modeling} eingesetzt werden?}
Vor allem bei Psychomotorischen und interpersonalen F"ahigkeiten.

\subsubsection{Was muss beim Einsatz von Computer-basierten Trainings beachtet werden?}
Es sollte ausreichend Struktur vorhanden sein. Die Lerner sollten Hilfe dabei kriegen, \emph{was} und \emph{wie} gelernt werden soll. F"ur Lerner, die in der Lage sind, mehr Kontrolle zu "ubernehmen, sollten adaptive Programme entwickelt werden, die ihnen mehr Kontrolle erlauben.

\subsubsection{Welche Faktoren wirken sich f"orderlich auf den Transfer des Gelernten aus?}
\begin{itemize}
        \item Zeit und Gelegenheit, das Gelernte anzuwenden
        \item Unterst"utzung und Ermunterung seitens der Vorgesetzten
        \item \emph{Debriefs} (Nachbesprechung) des Gelernten und wie es mit dem Job zusammenh"angt
        \item Zugang zu Datenbanken, Wissensbest"anden oder \emph{Communities of Practise}
\end{itemize}

\subsubsection{Was sind die Implikationen f"ur Leaders und Policymakers?}
\begin{itemize}
        \item Sicherstellen, dass Training die Bedarfe der Belegschaft deckt
        \item Maximierung der Art von Training, welches im Job hilft
        \item Motivation zum Lernen f"ordern
        \item Technologie gekonnt einsetzten
        \item Kontinuierliches Lernen f"ordern
\end{itemize}
