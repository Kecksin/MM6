\subsection{Fragen und Antworten zum Text von \textcite{albarracin_empirical_2006}}
\subsubsection{Welche Variablen wurden erfasst bei Wissensgewinn, Motivationsver"anderung und Verhaltens"anderung?}
\begin{itemize}
        \item \textbf{Verhaltens"anderung}
                \begin{itemize}
                        \item H"aufigkeit von Kondombenutzung
                \end{itemize}
        \item \textbf{Motivationsver"anderung}
                \begin{itemize}
                        \item Normen
                        \item Einstellungen
                        \item Kontrolle
                        \item F"ahigkeiten
                        \item Intentionen
                \end{itemize}
        \item \textbf{Wissensgewinn}
                \begin{itemize}
                        \item Wissen "uber HIV und wie man sich sch"utzen kann 
                \end{itemize}
\end{itemize}

\subsubsection{Wie gro"s waren die Ver"anderungen von Wissen, Motivation und Verhalten unmittelbar nach der Intervention und mit einiger Zeitverz"ogerung?}
Motivationsver"anderung war gro"s unmittelbar nach der Intervention und nahm danach ab. Verhaltens"anderung war gering unmittelbar nach der Intervention und nahm danach zu. Wissensver"anderung war und blieb konstant hoch. 

\subsubsection{Wie kann es sein, dass die Motivation nachl"asst, doch die Verhaltens"anderung zunimmt?}
Es werden zwei Erkl"arungen vorgeschlagen: Entweder es kommt zu einer Automatisierung von Kondombenutzung, oder die Motivation zur Kondombenutzung springt auf den Partner "uber. In beiden F"allen kann es zu einer verringerten Motivation kommen.

\subsubsection{Auf welchen verschiedenen Theorien zur Verhaltens"anderung basieren die verschiedenen Interventionen? Was betonen die unterschiedlichen Theorien als wesentlich f"ur die Verhaltens"anderung?}
Es gibt 2 Gruppen von Theorien. Gruppe $1$ besteht aus \emph{Theorie des "uberlegten Verhaltens}, \emph{Theorie des geplanten Handelns}, \emph{sozial-kognitiven Theorie} und dem \emph{information-motivation-behavioural skills model},  welches die drei vorherigen Theorien vereint. Diese Gruppe betont \emph{Einstellungen}, \emph{Normen}, \emph{Wissen} und \emph{Fertigkeiten, ein Verhalten auszuf"uhren}.
Die zweite Gruppe beeinhaltet das \emph{Health-Belief-model} und die \emph{Protection-Potivation-Theory}. Diese Gruppe betont vor allem die \emph{Angst} vor Krankheit.

\subsubsection{Waren aktive oder passive Interventionen wirkungsvoller? Was war hilfreich innerhalb der aktiven und passiven Interventionen?}
Grunds"atzlich waren aktive Interventionen wirkungsvoller. Innerhalb der aktiven Strategien waren \emph{Informationsvermittlung}, Vermittlung von \emph{Verhaltensf"ahigkeiten} und \emph{Stimmungkontrolle bei Alkoholgenuss} hilfreich. Innerhalb der passiven Stragegien waren ebenfalls Vermitllung von \emph{Verhaltensf"ahigkeiten} und \emph{Einstellungsbezogene} Ma"snahmen hilfreich. Siehe auch Fig. \ref{fig:albarracin1}.
\begin{figure}[h!]
        \begin{center}
                \begin{tikzpicture}[big/.style={shape=circle,draw,text width=4cm,font=\large, text centered},
                        passiv/.style={shape=rectangle,draw,rounded corners,text centered,text width=4cm,font=\footnotesize,node distance=0.5cm,fill=red!20},
                        aktiv/.style={shape=rectangle,draw,rounded corners,text centered,text width=4cm,font=\footnotesize,node distance=0.5cm,fill=green!20}]
                        \node [big,green] at (-3,3) (aktiv) {Aktive Interventionen};
                        \node [big,red] at (+3,1) (passiv) {Passive Interventionen};
                        \node [aktiv,above =of aktiv] (info) {Information};
                        \node [aktiv,above =of info] (alkohol) {Stimmungskontrolle bei alkoholgenuss};
                        \node [aktiv,above =of alkohol] (skill1) {Verhaltensf"ahigkeiten};
                        \node [passiv,above =of passiv] (einst) {Einstellungen};
                        \node [passiv,above =of einst] (skill2) {Verhaltensf"ahigkeiten};
                        \draw [->] (-8,-1) -- node [auto]{Wirkungsgrad} (-8,8.5);
                \end{tikzpicture}
        \end{center}
        \caption{Befunde zu aktiven und passiven Interventionen. Aktive Interventionen waren passiven "uberlegen. Au"serdem zu sehen sind die Strategien innerhalb der aktiven und passiven Interventionen, die hilfreich sind.}
        \label{fig:albarracin1}
\end{figure}

\subsubsection{Was erwies sich als nicht effektiv? Von welchen zwei Theorien wurde dieses Element empfohlen?}
Als nicht effektiv erwies sich Angst. Empfohlen wurde Angst vom Health-Belief Model und vom Protection-Motivation Model.

\subsubsection{Was wirkte bei allen Populationen? Was war bei spezifischen Populationen hilfreich?}
\begin{itemize}
        \item \textbf{Bei allen wirkungsvoll:}
                \begin{itemize}
                        \item Einstellungsbezogene Ma"snahmen
                        \item wahrgenommene Kontrolle
                        \item Verhaltensf"ahigkeiten
                \end{itemize}
        \item \textbf{nur bei Teenagern:}
                \begin{itemize}
                        \item normative Argumente
                \end{itemize}
        \item \textbf{nur bei Frauen:}
                \begin{itemize}
                        \item interpersonale F"ahigkeiten (den Partner "uberzeugen)
                \end{itemize}
\end{itemize}

\subsubsection{Wer sollte eine Intervention durchf"uhren?}
Experten!
