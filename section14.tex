\section{Theorie-Praxis-Austausch}
\subsection{Fragen und Antworten zum Text von \textcite{meyers_practical_2012}}
\subsubsection{Mit welchem Hauptspiel wurde das Quality Implementation Tool (QIT) entwickelt?}
Um Stakeholdern zu helfen, Programme zu implementieren.

\subsubsection{Auf welcher Basis wurde das QIT entwickelt?}
Auf Basis des \emph{Quality Implementation Framework}, das aus $25$ einzelnen \emph{Implementation Frameworks} besteht.

\subsubsection{Wie wird \emph{quality implementation} in dem Artikel definiert?}
Als das Umsetzen einer Innovation in die Praxis, so dass alle erw"unschten Ergebnisse erreicht werden.

\subsubsection{Wie wird der Begriff der Innovation definiert? Auf welchen drei theoretischen Annahmen basiert diese Definition?}
Innovationen werden definiert als Ideen, Praktiken, Programme oder Technologien, die als neu wahrgenommen werden. Folgende drei Annahmen:
\begin{enumerate}
        \item Innovationen m"ussen \emph{wohldefiniert} sein und \emph{Standards f"ur die Implementation} (im Sinne von Kernkomponenten) spezifizieren.
        \item \emph{Monitoring und Evaluation} sind unverzichtbar. Werden die Standards eingehalten? Werden die erw"unschten Ergebnisse erreicht?
        \item \emph{Adapation} ist oft notwendig. Kernkomponenten einer Innovation d"urfen aber nicht modifiziert werden.
\end{enumerate}

\subsubsection{Warum sind Monitoring und Evaluation wichtig bei der Implementierung von Innovationen?}
Damit "uberpr"uft werden kann, ob die Standards eingehalten und die Ziele erreicht werden.

\subsubsection{Warum m"ussen \emph{non-essential components} einer Innovation h"aufig an einen Kontext adaptiert werden? Was sollte hierbei beachtet werden? Warum sollten \emph{core elements} nicht modifiziert werden?}
Adaptation von nicht-essenziellen Komponenten ist n"otig, um eine \emph{Passung mit dem Kontext} zu erreichen. Das bezieht sich auf Ressourcen, Bedarfe und Pr"aferenzen. Kernkoponenten d"urfen aber nicht ver"andert werden, weil sonst die \emph{Integrit"at} einer Innovation bedroht ist.

\subsubsection{Welche Rolle spielt die Kapazit"at einer Organisation oder Community f"ur die erfolgreiche Implementierung von Innovationen?}
Die Kenntnis der Kapazt"at ist zentral f"ur das Schlie"sen der L"ucke zwischen Forschung und Praxis. 

\subsubsection{Skizzieren sie die drei Systeme des \emph{interactive systems frameworks for dissemination and implementation (ISF)}! Was ist jeweils die Funktion des Systems?}
Das \emph{Synthese \& Translationssystem} soll wissenschaftliche, theoretische und praktische Informationen sammeln und in ein benutzerfreundliches Format "ubersetzen.\\

\noindent Das \emph{Delivery System} (z.B. Schule, Organisation) soll die Innovation in die Praxis umsetzen.\\

\noindent Dabei unterst"utzt wird das Delivery System vom \emph{Support System}, welches Innovationsspezifische (F"ahigkeiten, Wissen, Motivation) und generelle Kapazit"aten (Infrastruktur) bilden soll. (Siehe Fig. \ref{fig:meyers1})
\begin{figure}[hb!]
        \begin{center}
                \begin{tikzpicture}[system/.style={rectangle,draw,rounded corners,text centered,text width=4cm,minimum size=2cm}
                        ]
                        %die drei Systeme
                        \node [system] at (0,0) (synthesis) {Synthese \& Translation};
                        \node [system,above=of synthesis] (support) {Support: {\footnotesize \emph{generelle} and \emph{innovationsspezifische} Kapazit"atenbildung}};
                        \node [system,above=of support] (delivery) {Delivery: {\footnotesize Innovation in die Praxis umsetzen}};
                        %Implementation
                        \node [right=of delivery] (outcomes) {Outcomes};
                        \draw [->,very thick,red] (delivery) -- (outcomes);
                        \draw [->,very thick,red] (support) -- (delivery);
                \end{tikzpicture}
        \end{center}
        \caption{Bildliche Darstellung des \emph{Interacting Systems Framework} von Wandersman et al. (2008). Die roten Pfeile symbolisieren die Stellen, an denen Implementierung stattfindet.}
        \label{fig:meyers1}
\end{figure}

\subsubsection{In welchen Formaten k"onnen \emph{Synthese und Translation} benutzerfreundlich zusammengefasst werden?}
\emph{Handb"ucher}, \emph{Spreadsheets (Excel)}, \emph{Interventionen} oder \emph{Strategien}.

\subsubsection{Welche Arten von Kapazit"aten werden im \emph{Support System} unterschieden?}
\emph{Innovationsspezifische} und \emph{generelle} Kapazit"aten.

\subsubsection{F"ur welche drei Zwecke kann das QIT eingesetzt werden?}
\begin{itemize}
        \item Planung von Quality Implementation
        \item Echtzeit-Monitoring von Implementierung
        \item Evaluation, inwiefern Quality Implementation gelungen ist
\end{itemize}

\subsubsection{Was sind die Aufgaben von \emph{implementation teams}?}
Die Mitglieder der \emph{Delivery Systems} zu informieren, vorzubereiten und zu unterst"utzen.

\subsubsection{Warum ist es wichtig, Mitglieder aus dem Delivery System sowie aus der Community in das implementation team zu integrieren?}
Um lokales (kontextbezogenes), praktisches Wissen zu bekommen.

\subsubsection{Was ist die Rolle von sogenannten \emph{program champions}? Nennen Sie ein Beispiel!}
Ein Experte des lokalen Systems, der willens und in der Lage ist, f"ur die Teilnahme am Programm zu werben. Z.B. ein Polizeichef oder ein Schuldirektor.

\subsubsection{Durch welche Strategien kann die Nutzung von Innovationen gef"ordert werden?}
Durch Richtlinien, die das Klima so ver"andern, dass die Nutzung erleichtert/gef"ordert wird. Das beinhaltet oft, Anreize zu schaffen, die Innovation zu nutzen, bzw. negative Anreize, wenn sie nicht genutzt werden.

\subsubsection{Welche Schritte k"onnen unternommen werden, um m"ogliche Hindernisse, die einer Implementierung entgegenstehen, zu minimieren?}
Hier sind alle \emph{action steps} der zweiten Komponente {Foster supportive organizational/communitywide climate and conditions} zu nennen:
\begin{itemize}
        \item program champion
        \item Bedarf kommunizieren
        \item Vorteile kommunizieren
        \item Widerstand von Stakeholdern entgegenwirken
        \item Verantwortlichkeiten etablieren
        \item gemeinsame Entscheidungsfindung und Kommunikation 
        \item administrative Unterst"utzung
\end{itemize}

\subsubsection{Wie kann eine Bedarfsanalyse f"ur Training oder Technical Assistance durchgef"uhrt werden?}
Wandersmann et al. (2012) stellen hierf"ur vier Tools zur Verf"ugung:
\begin{itemize}
        \item Organisationsanalyse: Informationen "uber das Setting
        \item Aufgabenanalyse: Informationen "uber die Aufgaben, die zu erledigen sind
        \item Personenanalyse: Wie sind die Personen, die am Training teilnehmen?
        \item Werteanalyse: der Nutzen soll die Kosten "uberwiegen
\end{itemize}

\subsubsection{Welche Aspekte sollten bei der Evaluation der Effektivit"at der Implementierung erfasst werden?}
\begin{itemize}
        \item \emph{Fidelity} (oder auch: adherence): Wird die Innovation auch so umgesetzt, wie sie konzipiert wurde?
        \item \emph{Dosierung} (oder auch: Dauer): Wie viel von der Innovation wurde tats"achlich umgesetzt?
        \item \emph{Quality of delivery}: Wie gut, enthusiastisch, vorbereitet, etc. wurde die Innovation umgesetzt?
        \item \emph{Participant Responsiveness}: Inwiefern machen die Programmteilnehmer auch das, was zum Programm geh"ort?
        \item \emph{Program differentiation:} Wie sehr unterscheidet sich die Innovation von anderen Innovationen?
        \item \emph{Program reach:} Welcher Anteil der Zielpopulation nimmt auc wirklich teil?
        \item \emph{Document adaptations}
\end{itemize}

\subsubsection{Welches methodische Vorgehen ist sinnvoll, um \emph{implementation fidelity} zu messen?}
Innovation sollten Kernkomponenten beinhalten. F"ur diese Kernkomponenten m"ussen Indikatoren gefunden werden, die dann mit allem, was die psychometrische Trickkiste hergibt, gemessen werden k"onnen (Fragebogen, Checklisten, Interviews, etc.)
