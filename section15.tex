\section{Theorie-Praxis-Austausch}
\subsection{Fragen und Antworten zum Text von \textcite{stephan_bridging_2006}}
\subsubsection{Stephan beschreibt die Beziehung zwischen Theorie und Praxis als Intergruppenbeziehung von Gruppen mit unterschiedlichem kulturellen Hintergrund. Durch welche weiteren Aspekte ist die Beziehung gekennzeichnet?}
\begin{itemize}
        \item Expertise in unterschiedlichen Bereichen
        \item geringes Wissen "uber die andere Gruppe
        \item Mitunter Geringsch"atzung der anderen Gruppe
\end{itemize}

\subsubsection{Was kennzeichnet die Perspektive der Praxis im Bezug auf \emph{intergroup relation programs}?}
\begin{itemize}
        \item Techniken zur Ver"anderung von Gef"uhlen, Gedanken, Verhalten
        \item praktische Probleme wie Implementierung, Sequenzierung und direkte Reaktion
        \item Sensibilit"at f"ur sozialen Kontext und Kommunikation
        \item Effektivit"at im Fokus
        \item Beurteilung aufgrund eigener Erfahrungen
        \item Subjektivit"at
        \item \emph{Experiental Learning} anstatt experimentelles Design
\end{itemize}

\subsubsection{Was ist mit \emph{experiental learning} gemeint? Kennen Sie diesen Begriff aus der P"adagogischen Psychologie?}
Ich kenne diesen Begriff nicht. Trotzdem denke ich zu wissen, was damit gemeint ist, n"amlich etwas, was man auch als ``leaning by doing'' bezeichnen k"onnte. Man macht und probiert verschiedenes aus.

\subsubsection{Was kennzeichnet demgegen"uber die Perspektive der Wissenschaft auf \emph{intergroup relation programs}?}
\begin{itemize}
        \item Gef"uhle werden eher studiert als gezeigt
        \item nicht praktische, sondern theoretische Fragen im Fokus
        \item Erfolg ist nicht, wie sehr Ergebnisse in Praxis einflie"sen
        \item nicht subjektive, sondern m"oglichst objektive Schl"usse
        \item Experimentelles Design und quantitative Analysen
\end{itemize}

\subsubsection{Was folgt aus diesen Perspektiven im Hinblick auf das Interessen an den f"unf von Stephan diskutierten Aspekten von \emph{inergroup relation programs}?}
Siehe Fig. \ref{fig:stephan1}
\begin{figure}[hb!]
        \begin{center}
                \begin{tikzpicture}[every node/.style={shape=rectangle,rounded corners,text centered,text width=2.2cm,minimum size=1.5cm,}
                        ]
                        \node [fill=magenta!20]at (-6,0) (goals) {Ziele der Programme};
                        \node [fill=magenta!20,right=of goals] (groups) {Zielgruppen};
                        \node [fill=magenta!20,right=of groups] (program) {Programm und Techniken};
                        \node [fill=cyan!20,right=of program] (processes) {Psychologische Prozesse};
                        \node [fill=cyan!20,right=of processes] (outcomes) {Ergebnisse};
                        \draw [->] (-7,-1) -- (9,-1);
                \end{tikzpicture}
        \end{center}
        \caption{Die $5$ Aspekte von intergroup relation programs nach \textcite{stephan_bridging_2006}, angeordnet auf einem Kontinuum von \textcolor{magenta}{Praktikertum} (links) nach \textcolor{cyan}{Forschertum} (rechts). Auch Forscher interessieren sich etwas f"ur die Aspekte links auf dem Kontinuum, jedoch nicht so stark.}
        \label{fig:stephan1}
\end{figure}

\subsubsection{Aus welchen Gr"unden kann es Kommunikationsprobleme zwischen Praktikerinnen und Wissenschaftlerinnen geben? Nennen Sie ein Beispiel aus dem Artikel!}
Die zwei Hauptgr"unde sind:
\begin{itemize}
        \item zu wenig Wissen
        \item kein Kontakt
\end{itemize}

Au"serdem sprechen beide nicht die gleiche Sprache. Sie benutzen zwar unter Umst"anden die gleichen Begriffe, diese Begriffe haben aber jeweils eine andere Bedeutung (Mediation, Prozess).

\subsubsection{Inwieweit k"onnen beide Gruppen voneinander profitieren?}
Praktiker k"onnen \emph{Hypothesen oder Theorien liefern}, auf die Wissenschaftler noch nicht gekommen sind. Die Programme der Praktiker k"onnen zur Testung dieser Hypothesen genutzt werden.\\
Wissenschaftler k"onnen Praktiker "uber die \emph{differentielle Effektivit"at} von Techniken bezogen auf bestimmte Zielgruppen informieren. Sie k"onnen ihnen au"serdem helfen, die Effektivit"at einer Ma"snahme \emph{objektiv einzusch"atzen}. Abgesehen davon k"onnen sie nat"urlich auf \emph{bereits vorhandene Befunde} hinweisen.

\subsubsection{Nennen Sie drei Beispiele f"ur Strategien zur F"orderung des Kontaktes von Wissenschaft und Praxis!}
\begin{itemize}
        \item Journals der anderen Gruppe lesen
        \item f"ur Journals der anderen Gruppe schreiben
        \item Workshops der anderen Gruppe besuchen
\end{itemize}

\subsubsection{Was sind m"ogliche Hinderungsgr"unde f"ur die Umsetzung der genannten Strategien? Z.B. was spricht aus der Sicht eines Wissenschaftlers dagegen, einen Artikel in einer Praxiszeitschrift zu ver"offentlichen?}
Praxiszeitschriften haben vermutlich kein gutes Ansehen in Wissenschaftskreisen. Ein Wissenschaftler wird deshalb versuchen, in einem m"oglichst renommierten Journal zu ver"offentlichen.
