\section{Grundlagen vom Pr"avention und Intervention}

\subsection{Fragen und Antworten zum Text von \textcite{finifter_comprehensive_2005}}
\subsubsection{Was sind die 4 Hauptprobleme bei g"angigen Herangehensweisen an Bedarfsanalysen?}
\begin{itemize}
        \item Bedarfsanalyse basiert auf Intuition, ``gesundem Menschenverstand'' oder anekdotischer Evidenz
        \item es wird nur eine Variable an einem kleinen Teil der Population erhoben
        \item emphohlene L"osungen werden nicht implementiert
        \item Forscher und in der Praxis t"atige Personen sind voneinander isoliert
\end{itemize}

\subsubsection{Was schlagen die Autoren und Autorinnen als \emph{best practices} vor?}
\begin{itemize}
        \item empirische Bedarfsanalyse
        \item vielschichtige Erfassung der Bedarfe
        \item Verbreitung und Implementierung (Handlungsorientierung)
        \item Kollaboration von Forschenden und in der Praxis t"atigen Personen
\end{itemize}

\subsubsection{Welche Komponenten umfasst der Prozess im vorgeschlagenen Modell und was sollte bei der ersten Komponente beachtet werden?}
Die 3 Komponenten sind:
\begin{itemize}
        \item Bedarfsanalyse
        \item Verbreitung
        \item Implementierung
\end{itemize}

\noindent Die erste Komponente, die Bedarfsanalyse, sollte empirisch fundiert und vielschichtig sein. (Siehe auch Fig. \ref{fig:finifter1})

\begin{figure}[hb!]
        \begin{center}
                \begin{tikzpicture}[every node/.style={shape=ellipse,draw,text width=3cm,text centered}]
                        \node at (0,0) (assess) {Bedarfsanalyse};
                        \node [right=of assess] (dissem) {Verbreitung};
                        \node [right=of dissem] (implement) {Implementierung};
                        \draw [->] (assess) -- (dissem);
                        \draw [->] (dissem) -- (implement);
                \end{tikzpicture}
        \end{center}
        \caption{Modell zur Bedarfsanalyse nach \textcite{finifter_comprehensive_2005}}
        \label{fig:finifter1}
\end{figure}

\subsubsection{\Textcite{finifter_comprehensive_2005} beschreiben ein Beizpiel zur Umsetzung. Was waren Ziele und Zielgruppe?}
Zielgruppe waren "altere Menschen ab 60. die Ziele waren:
\begin{itemize}
        \item F"orderung von Unabh"angigkeit
        \item Defragmentarisierung der Hilfen
        \item vollst"andige Erf"ullung der Bedarfe von "alteren Menschen (durch Verbindung von Forschung und Praxis)
\end{itemize}

\subsubsection{Welche Methoden wurden f"ur die Datenerhebung und die anderen beiden Komponenten benutzt?}
Methoden f"ur die Datenerhebung:
\label{sec:verbreit}
\begin{itemize}
        \item Zensusdaten
        \item Daten von Gesundheitsorganisationen
        \item Telefoninterviews mit "alteren Menschen
        \item Fokusgruppen
        \item Fragebogen
                \begin{itemize}
                        \item Pfleger aus Familie oder Freundeskreis
                        \item Hilfsorganisationen f"ur "altere Menschen
                        \item Gesundheitsorganisationen
                        \item religi"ose Organisationen
                        \item Schl"usselpersonen der Community
                \end{itemize}
\end{itemize}

\noindent Methoden der Verbreitung:
\begin{itemize}
        \item Gr"undung eines Hilfskommitees
        \item Konferenz zur Verbreitung der Bedarfe
        \item Webseite
        \item Vortr"age auf Fachkonferenzen
        \item Journal Artikel und Abschlussbericht
\end{itemize}

\noindent Zur Implementation wurde die \emph{senior services coalition} gegr"undet. Die dringendsten Massnahmen waren:
\begin{itemize}
        \item Transport 
        \item Dienstleistungen bei "alteren Menschen zu Hause
        \item Zugang zu Informationen
\end{itemize}

\subsubsection{Was waren m"ogliche Einschr"ankungen der beschriebenen Herangehensweise?}
\begin{itemize}
        \item Stichprobenfehler (z.B. "altere Menschen ohne Telefon wurden nicht ber"ucksichtigt)
        \item Fehlende Daten durch Nichtbeantwortung einiger Fragen
        \item Selbstselektionstendenz bei Befragungen 
        \item evtl. h"atten andere, nicht befragte Gruppen noch wertvolle Informationen liefern k"onnen
\end{itemize}

\subsubsection{Weswegen wurde eine Fallstudie zur Alzheimer Erkrankung erstellt? Wie wurde vorgegangen?}
Die Fallstudie diente als eine \emph{baseline} f"ur Informationen "uber das gegenw"artige System von Hilfen f"ur Familien, die von Alzheimer betroffen sind. Warum ausgerechnet die Alzheimer Erkrankung? 5 Gr"unde:
\begin{itemize}
        \item AD hat gesellschaftliche eine gro"se Relevanz
        \item AD hat physische, emotionale und finanzielle Wirkungen und kennt keine sozio"okonomischen Grenzen
        \item mit Fortschreiten der Erkrankung ben"otigen Patienten alle Dienste, die es gibt
        \item verschiedene Experten konnten die Fallstudie erstellen und testen
        \item es ist machbar und hilfreich, die medizinischen und sozialen Systeme zu erfassen, die f"ur die Erf"ullung der entstehenden Bedarfe von Betroffenen notwendig sind
\end{itemize}

\subsubsection{Was waren die St"arken und Schw"achen der Community, die durch die Mehrebenenbedarfsanalyse herausgearbeitet wurden?}
St"arken:
\begin{itemize}
        \item allgemeine Zufriedenheit mit den Diensten der Community
        \item relative gute Gesundheit der Community Mitglieder
\end{itemize}

\noindent Schw"achen:
\begin{itemize}
        \item zuverl"assige Informationsquellen sind schwer zu finden
        \item Senioren mit niedrigem und mittlerem Einkommen k"onnen sich manche Dienste nicht leisten
        \item Gesundheitsorganisationen und Pflegepersonen wollen Programme um zu lernen, wie ihre Hilfeleistungen besser werden k"onnen
        \item Transportm"oglichkeiten sind\ldots
                \begin{itemize}
                        \item zu teuer
                        \item zu unflexibel
                        \item unzuverl"assig
                        \item f"ur Behinderte nicht geeignet
                \end{itemize}
        \item Handwerker sind schwer zu kriegen und zu teuer
        \item Medikamente sind zu teuer
\end{itemize}

\subsubsection{Wie wurden die Ergebnisse verbreitet? Wie wurden die herausgearbeiteten Ergebnisse umgesetzt?}

Zur Verbreitung siehe Frage \ref{sec:verbreit}. Zur Umsetzung: 
\begin{itemize}
        \item Bildung eines Hilfskommitees, das 3 wiederum drei Subkommitees bildete
                \begin{itemize}
                        \item Transport
                                \begin{itemize}
                                        \item alle Transportdienste wurden an einen Tisch geholt
                                        \item es wird daran gearbeitet, dass es nur noch eine Telefonnummer gibt, die dann alle anrufen k"onnen
                                \end{itemize}
                        \item Dienste bei Senioren zu Hause
                        \item Zugang zu Informationen
                \end{itemize}
        \item Geriatrie-Fortbildungen f"ur "Arzte
\end{itemize}


