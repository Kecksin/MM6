\section{Diversit"at}
\subsection{Fragen und Antworten zum Text von \textcite{stahlberg_geschlechterdiskriminierung_2009}}
\subsubsection{Welche normativen und "okonomischen Probleme ergeben sich aus der Benachteiligung von Frauen?}
Ein normatives Problem ergibt sich, wenn eine Gesellschaft sich Gleichberechtigung auf die Fahne schreibt, die dann aber nicht gegeben ist. Ein "okonomisches Problem ist es, wenn Geld daf"ur ausgegeben wird, Frauen auszubilden, diese Investition aber nicht wieder reingeholt wird, weil Frauen nicht in Jobs kommen.

\subsubsection{Welche Aspekte umfasst der Oberbegriff Geschlechtsstereotyp?}
\begin{itemize}
        \item Pers"onlichkeitseigenschaften
        \item Rollenverhalten
        \item kognitive F"ahigkeiten
        \item physische Merkmale
\end{itemize}

\subsubsection{Wodurch unterscheiden sich deskriptive und pr"askriptive Geschlechtsstereotype?}
\emph{Deskriptiv} ist eine Beschreibung dessen, wie sich M"anner oder Frauen tats"achlich verhalten. \emph{Pr"askriptiv} meint, wie sich verhalten sollen.

\subsubsection{Welche Begriffe beschreiben die in weiblichen und m"annlichen Stereotypen enthaltenen Eigenschaften?}
Weiblich:
\begin{itemize}
        \item Warmherzigkeit
        \item Communion
        \item Expressivit"at
\end{itemize}

\noindent M"annlich:
\begin{itemize}
        \item Kompetenz und Rationalit"at
        \item Agency
        \item Instumentalit"at
\end{itemize}

\subsubsection{Was ist moderner, subtiler oder wohlmeinender Sexismus und weshalb ist er problematisch?}
Sie sind nicht so leicht als Sexismus zu enttarnen, tragen aber zur Aufrechterhaltung von Diskrimierung bei.

\subsubsection{Welche Unterschiede finden sich zwischen den Geschlechtern?}
Keine Unterschiede bei:
\begin{itemize}
        \item Intelligenz
        \item Motiven
        \item mathematische F"ahigkeiten
        \item Anzahl der Studienanf"anger
        \item Anzahl der Studienabsolventen
\end{itemize}

\noindent Unterschiede bei:
\begin{itemize}
        \item verbale F"ahigkeiten: Frauen besser (aber nur wenig)
        \item "altere Generation: Frauen haben geringeres Bildungsniveau
        \item j"ungere Generation: Frauen haben h"oheres Bildungsniveau
        \item Gehalt: Frauen verdienen ca. 15\% weniger
        \item Beruf: Frauen seltener in F"uhrungspositionen
        \item weniger Frauen bei den Professoren
\end{itemize}

\subsubsection{Wie erfolgt geschlechtstypisierende Darstellung in den Medien?}
\begin{itemize}
        \item weniger Frauen in Hauptrollen
        \item Betonung der stereotypischen Rollen (Betonung von Attraktivit"at bei Frauen)
        \item K"orperhaltung
        \item \emph{Generisches Makulinum}
        \item \emph{Linguistic Expectancy Bias}
\end{itemize}
\subsubsection{Was sind \emph{generisches Maskulinum} und \emph{Linguistuc Expectancy Bias} und wie wirken sie sich aus?}
\begin{description}
        \item[Generisches Maskulinum] Wenn immer nur die maskuline Sprachform verwendet wird.
        \item[Linguistic Expectancy Bias] Stereotyp-kongruentes Verhalten wird abstrakter beschrieben als stereotyp-inkongruentes Verhalten. Z.B. wird "uber positive Ergebnisse von M"annern in F"uhrungspositionen eher etwas allgemeines gesagt wie ``Er kann andere mitreissen.'' "Uber Frauen wird dagegen eher etwas gesagt wie ``Ihr gelang es, Mitglieder aus dem Team zum mitmachen zu bewegen.'' Bei negativen Ergebnissen w"are es dagegen genau anders herum.
\end{description}
\subsubsection{Verschiedene Prozesse und Effekte tragen zur Geschlechterdiskriminierung und Verringerung der Erfolgschancen von Frauen bei. Welche sind das und wie wirken sie sich aus?}
\begin{description}
        \item[Stereotype Threat.] In einer Situation, in der ein Stereotyp aktiviert ist, kann der Druck, der durch dieses Stereotyp erzeugt wird, dazu f"uhren, dass Personen der stereotypisierten Gruppen erwarten, sich im Sinne des Stereotyps zu verhalten. Es kann auch dazu f"uhren, dass sie sich tats"achlich so verhalten. 
        \item [Lack-of-Fit.] Wahrgenommene mangelnde Passung zwischen Eigenschaften einer Person und Anforderung des Berufs. Erfolgserwartungen an diese Person im Beruf sind schlechter, je schlechter die Passung. 
        \item [Think-Manager-Think-Male.] Ein Spezialfall des Lack-of-Fit Modells. Der Managerberuf wird als typisch m"annlich angesehen und Frauen wird nicht zugetraut, diesen Beruf erfolgreich auszu"uben. W"ahrend in den $70$-er Jahren noch beide Geschlechter von diesem Ph"anomen betroffen waren, sind es mittlerweile nur noch die M"anner, die den Frauen den Managerberuf nicht zutrauen.
        \item [Glass Ceiling.] Eine unsichtbare und un"uberwindbare Barriere f"ur Frauen, die Karriere machen wollen.
        \item [Glass Escalator.] Eine unsichtbare Hilfe f"ur M"anner, die Karriere machen wollen. 
        \item [Token-Effek.t] Tokens sind Frauen und M"anner in geschlechteruntypischen Berufen. Der Effekt beschreibt das Verhalten der Mehrheit den Tokens gegen"uber. Sie erfahren mehr Aufmerksamkeit und damit evtl. mehr Druck (Stereotype Threat), Unterschiede zwischen Tokens und Mehrheit werden "uber- und Gemeinsamkeiten untersch"atzt und Tokens werden als typische Vertreter ihrer Gruppe angesehen.
        \item [Queen-Bee Syndrom.] Frauen, die in M"annerberufen erfolgreich sind, stimmen Geschlechtsstereotypen zu. 
        \item [Sex-Role Spillover.] Das "Ubertragen von Geschlechtsrollen auf den beruflichen Kontext. Tritt vor allem bei Tokens auf und ist h"aufiger bei Frauen als bei M"annern, weil es weniger M"anner in typischen Frauenberufen gibt. 
        \item [Backlash-Effekt.] Von Frauen und von M"annern wird geschlechtstypisches Verhalten erwartet. Im Beruf wird aber manchmal Verhalten erwartet, dass der Geschlechtsrolle widerspricht. Das kann zum Backlash-Effekt f"uhren. So kann es Frauen, die sich typisch m"annlich verhalten, passieren, dass sich das negativ f"ur sie auswirkt. Ebenso kann das M"annern Passieren, die sich typisch weiblich verhalten.
\end{description}
