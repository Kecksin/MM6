\subsection{Fragen und Antworten zum Text von \textcite{bolger_diary_2003}}
\subsubsection{Was versteht man unter der Diary Methode?}
Menschen berichten regelm"aßig "uber die Erfahrungen und Ereignisse des t"aglichen Lebens. Es werden Einzelheiten erfasst, die mit bisherigen Methoden nicht erfasst werden k"onnen. 

\subsubsection{F"ur welche Forschungsfragen ist die Diary Methode besonders geeignet?}
\begin{itemize}
        \item \textbf{Aggregating over time:} Wie ist die typische Person und wie sehr unterscheiden sich Menschen voneinander?
        \item \textbf{Modeling the time course:} Wie ver"andert sich die typische Person "uber die Zeit und wie unterscheiden sich Menschen in dieser Ver"anderung?
        \item \textbf{Modeling within-person processes:} Wie l"auft ein Prozess innerhalb der typischen Person ab und wie unterscheiden sich Menschen in Prozessen?
\end{itemize}

\subsubsection{Was sind Vor- und Nachteile der Verwendung der Diary Methode?}
\begin{itemize}
        \item \textbf{Vorteile}
                \begin{itemize}
                        \item nat"urlicher Kontext
                        \item Informationen komplement"ar zu traditionellen Designs
                        \item keine Retrospektion
                        \item Sch"atzung intrapersonaler Tendenzen
                        \item Zeitverlauf
                \end{itemize}
        \item \textbf{Nachteile}
                \begin{itemize}
                        \item meist nicht experimentell, kausale Schl"usse nicht m"oglich
                        \item nicht jedes relevante Ereignis wird identifiziert
                        \item "Ubergeneralisierung von nicht relevaten Ereigenissen
                        \item Training n"otig
                        \item kaum Kenntnisse "uber Effekte des Tagebuchf"uhrens
                \end{itemize}
\end{itemize}

\subsubsection{Welche unterschiedlichen M"oglichkeiten der technischen Umsetzung gibt es und was sind jeweils die Vor- und Nachteile?}
\paragraph{Papier und Bleistift.}
Immer noch am h"aufigsten verwendet.
\begin{itemize}
        \item \textbf{Vorteile:}
                \begin{itemize}
                        \item einfach
                \end{itemize}
        \item \textbf{Nachteile:}
                \begin{itemize}
                        \item Vpn vergessen, Tagebuch zu f"uhren
                        \item Fehler in der Retrospektion
                        \item \emph{Uncertain Compliance} in Bezuf auf Anzahl und Validit"at der Eintr"age
                        \item keine Informationen zu Reationszeiten
                \end{itemize}
\end{itemize}

\paragraph{Erweiterte Papiertageb"ucher}
Antworten werden zwar immer noch mit P\&P Frageb"ogen gesammelt, aber durch \emph{Signalisierungsger"ate} erg"anzt: Pager, vorprogrammierte Armbanduhren, Telefonanruf\ldots
\begin{itemize}
        \item \textbf{Vorteile:}
                \begin{itemize}
                        \item L"osung des Problems des Vergessens
                        \item Erleichterung bei den anderen zwei Problemen von P\&P
                        \item Einfachheit von P\&P bleibt erhalten
                \end{itemize}
        \item \textbf{Nachteile:}
                \begin{itemize}
                        \item trotzdem noch Retrospektionsfehler
                        \item aufw"andig und kostspielig
                        \item \emph{Disruuption}
                \end{itemize}
\end{itemize}

\paragraph{Elektronische Datensammlung mit Handger"aten}
\begin{itemize}
        \item \textbf{Vorteile:}
                \begin{itemize}
                        \item erm"oglichen Signalisierung
                        \item Zeit- und Datumstempel
                        \item flexible Pr"asentation der Fragen
                        \item Datenerfassung und Genauigkeit
                        \item Ber"ucksichtigung der Zeitpl"ane der Vpn
                \end{itemize}
        \item \textbf{Nachteile:}
                \begin{itemize}
                        \item hohe Entwicklungskosten
                        \item g"unstige Ger"ate (noch) mangelhaft
                        \item Kosten- und Ressorcenintensives Training der Vpn n"otig
                        \item Risiko einer ``digitalen Kluft''
                \end{itemize}
\end{itemize}

\paragraph{Neue M"oglichkeiten.}
\begin{itemize}
        \item \textbf{Improving self-report technology}
                \begin{itemize}
                        \item mobile Kommunikation erm"oglicht online und interaktiven Kontakt mit Vpn
                        \item Fortschritte bei Spracherkennung 
                \end{itemize}
        \item \textbf{Integration of collateral information} z.B. Herzfrequenz, etc.
\end{itemize}

\subsubsection{Was sollte bei der Datenauswertung beachtet werden?}
\begin{itemize}
        \item intrapersonelle Messungen nicht unabh"angig
        \item benachbarte Eintr"age "ahnlicher als entfernte (Problen der Autokorrelation)
        \item gro"se und auch personenabh"angig unterschiedliche Anzahl an Berichten (herk"ommliche Messwiederholungsmethoden nicht m"oglich)
        \item zeitliche Muster m"ussen durch entsprechende Methiden modeliert werden
\end{itemize}

Generell ist zu sagen, dass \emph{Hierarchische Lineare Modelle} gut geeignet sind, um solche Daten auszuwerten. Spezielle Auswertungsmethoden bezogen auf die drei unterschiedlichen Forschungsfragen sind:
\begin{itemize}
        \item Einfaktorielle ANOVA mit Random Effects (Aggregating over time: What is the typical person like and how much do peopke differ?) 
        \item Linear Growth Model (Modeling the time course: How does a typical person change over time and how do people differ?)
        \item Random Coefficients Regression Model (What is the within-person process of the typical person and how do people differ?)
\end{itemize}


