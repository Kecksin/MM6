\subsection{Fragen und Antworten zum Text von \textcite{chen_protective_2012}}

\subsubsection{Welche Faktoren auf welchen Ebenen tragen zu gesundheitlichen Ungleichheiten basierend auf  dem sozio"okonomischen Status bei (health disparities)?}
Folgende Faktoren treten auf individueller und Familien- Nachbarschaftsebene auf:
\begin{itemize}
        \item psychosoziale Faktoren:
                \begin{itemize}
                        \item Stress
                        \item negative Emotionen
                        \item schlechte Gesundheitsverhaltensweisen
                \end{itemize}
        \item physikalische Umweltfaktoren: 
                \begin{itemize}
                        \item nachbarschaftliche Umgebungsbebauung
                        \item Umwelteinfl"usse, die die Gesundheit beeintr"achtigen
                \end{itemize} 
\end{itemize}


\subsubsection{Was ist Resilienz und was tr"agt dazu bei auf den Ebenen \emph{des Individuums} \emph{der Familie} und \emph{der Nachbarschaft}? }
Resilienz ist die Anpassung im Angesicht der Bedrohung. Die Literatur identifiziert wichtige Faktoren auf den drei genannten Ebenen. Diese Faktoren wirken als Puffer bei Widrigkeiten und sch"utzen Kinder vor Verhaltensauff"alligkeiten und schulischem Misserfolg.
\begin{itemize}
        \item \textbf{Indivuduum}
                \begin{itemize}
                        \item Temperament
                        \item kognitive Faktoren
                \end{itemize}
        \item \textbf{Familie}
                \begin{itemize}
                        \item elterliche W"arme
                        \item psychische Gesundheit der Eltern
                \end{itemize}
        \item \textbf{Nachbarschaft}
                \begin{itemize}
                        \item positive Schulumgebung
                \end{itemize}
\end{itemize}

\subsubsection{Was ist mit Verlagern (\emph{shifting}) und Beharren (\emph{persisting}) in dem von der Autorin vorgeschlagenen Modell gemeint? }
\emph{Shifting/Verlagern} meint, sich selbst auf die "au�eren Umst"ande einzustellen:
\begin{itemize}
        \item Akzeptanz von Stressoren
        \item positivere Bewertung stressvoller Situationen 
        \item effektive Emotionsregulation in stressvollen Situationen 
\end{itemize}

\emph{Persisting/Beharren} meint, die erfolgreiche Anpassung an dauerhafte Widrigkeiten. Diese Anpassung hilft dabei, in schwierigen Situationen Sinn zu finden und den Optimismus aufrechtzuerhalten:
\begin{itemize}
        \item die Not durch St"arke durchstehen 
        \item Bedeutung finden 
        \item Optimismus, die eigene Zukunft betreffend
\end{itemize}

WICHTIG: Nicht die einzelnen Strategien (shifting oder persisting) wirken sich positiv auf physiologische Reaktionen von Low-SES-Individuen aus, sondern die Kombination beider Strategien.

\subsubsection{Wie k"onnen sich wahrgenommener Sinn des Lebens und Optimismus auf die k"orperliche Gesundheit auswirken?}
Individuen mit niedrigem SES und einem h"oheren Empfinden des Lebens als sinnvoll hatten \emph{niedrigere Konzentration von IL6} (Entz"undungsmarker, der mit Herz-Kreislauferkrankungen in Verbindung gebracht wird). Im Gegensatz dazu war wahrgenommener Sinn des Lebens nicht mit dem IL6-Marker von Individuen mit hohem SES assoziiert.

Individuen mit niedrigem SES und h"oheren Pessimismuswerten (Gegenteil von Optimismus) hatten einen \emph{h"oheren ambulanten Blutdruck} und eine h"ohere Wahrscheinlichkeit f"ur Bluthochdruck als Low-SES Individuen mit niedrigen Pessimismuswerten oder High-SES-Individuen.


\subsubsection{Ist die Kombination der Strategien Verlagern (\emph{shifting}) und Beharren (\emph{persisting}) f"ur Personen mit h"oherem sozio"okonomischen Status ebenso geeignet?}
Nein. Bei High-SES Individuen k"onnen \emph{proaktive Bew"altigungsstrategien}, die auf die Beseitigung von Stressoren gerichtet sind, effektiver sein, da diese im Schnitt mehr Ressourcen besitzen. 

\subsubsection{Chen (2012) beschreibt zwei Studien, in denen sie und ihre Kolleginnen zeigen konnten, wie sich die Kombination von Verlagern (\emph{shifting}) und Beharren (\emph{persisting}) bei niedrigem sozio"okonomischen Status positiv auf die Gesundheit auswirken. Was genau haben sie bei wem gemessen und welche Zusammenh"ange konnten sie finden?}
In einer Stichprobe Erwachsener, wurden die Umst"ande von deren Kindheit beurteilt. Basierend auf 24 Messungen des Risikos "uber sieben physiologische Systeme, wurde ein Index des kumulativen physiologischen Risikos (\emph{allostatic load}) entwickelt. Zur Vorhersage des allostatic load (der Erwachsenen) interagierte der Einsatz von shift-and-persist Strategien (erfasst "uber die Abfrage von Coping-Stilen und Zukunftsorientierung) mit Low-SES-Kindheiten. 

Unter diesen Personen war die Kombination von hohem Verlagern (shifting) und hohem Beharren (persist) mit den niedrigsten allostatic load-Werten assoziiert. Dieser Zusammenhang fand sich nicht bei High-SES-Individuen.\\

Die Effekte von shift-and-persist wurden zudem in einer klinischen Stichprobe mit Kindern, die unter Asthma leiden untersucht. Hier zeigte sich, dass Low-SES-Kinder mit hohen shift-and-persist Werten 
\begin{itemize}
        \item geringere Asthma-Entz"undungswerte aufwiesen 
        \item h"ohere shift-and-persist Werte (sechs Monate sp"ater) 
\end{itemize}
weniger Asthma Beeintr"achtigungen vorhersagen (geringere Schulfehlzeiten, weniger Rettungs-Inhalator Nutzung). Dieser Zusammenhang fand sich nicht bei High-SES-Kindern. Low-SES-Kinder mit hohen shift-and-persist Werten waren 
in ihren klinischen Profilen High-SES-Kindern "ahnlicher, als Low-SES-Kindern mit niedrigen shift-and-persist Werten. 

\subsubsection{Wodurch k"onnen Kinder die Kombination der Strategien Verlagern (\emph{shifting}) und Beharren (\emph{persisting}) lernen? Beschreiben Sie die Studie, mit der Chen und Kolleginnen ihre Hypothesen empirisch untermauern konnten.}
Durch \emph{Vorbilder (role models)} in der fr"uhen Kindheit.\\

Chen und Kollegen f"uhrten Interviews mit gesunden Jugendlichen "uber ihre Vorbilder, wobei sie folgendes bewerteten: 
\begin{itemize}
        \item SES (sozio"okonomischer Status) 
        \item Einsatz von shift-and-persist Strategien 
        \item Indikatoren des Kardiovaskul"aren Risikos (systemische Entz"undungen und Cholesterinwerte) 
\end{itemize}

In dem Ma�e in dem Low-SES-Jugendliche von unterst"utzenden und inspirierenden Vorbildern berichteten, stieg auch deren Einsatz von shift-and-persist Strategien. (High-SES-Individuen zeigten kein solches Muster.) Zudem hing bei Low-SES-Jugendlichen ein h"oherer shift-and-persist Einsatz mit niedrigeren IL6 Werten und niedrigeren Cholesterinwerten zusammen.





























