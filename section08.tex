\subsection{Fragen und Antworten zum Text von \textcite{kohfeldt_five_2012-1}}
\subsubsection{PAR (Participatory Action Research) beabsichtigt, angeblich unver"anderliche Ideologien durch demokratisierte Wissensproduktion und Empowerment in Frage zu stellen. Wie wird der Bedarf an diesem Prozess erkl"art, bzw. welchen Missst"anden soll damit begegnet werden?}
Die Art, in der wir "uber soziale Probleme nachdenken, hat Einfluss darauf, wen wir f"ur diese Probleme verantwortlich machen und damit auch eventuelle L"osungsans"atze. Diese Sichtweise ist h"aufig \emph{defizitorientiert} und sucht den Fehler bei den Benachteiligten.\\
Au"serdem wird die Weltsicht, die durch Medien und Politik propagiert wird nicht als \emph{eine}, sondern als \emph{die einzig m"ogliche} Sichtweise wahrgenommen.

\subsubsection{Was ist gemeint mit dem Begriff \emph{second-order change} und wie grenzt sich dieser ab vom Begriff \emph{first-order change}?}
\begin{itemize}
        \item \textbf{second-order change.} Restrukturieung des Systems (Role Relations)
        \item \textbf{first-order change.} Grundstrukturen des Systems bleiben erhalten
\end{itemize}

Second-order change wird wird wahrscheinlicher, wenn Problemdefinitionen auf \emph{strukturelle} Formulierungen fokussieren.

\subsubsection{Wie funktioniert die \emph{Five-Whys-Methode}? Zu welchem Zweck kann sie eingesetzt werden?}
Entwickelt von Sakichi Toyoda, f"ur wirtschaftliche Zwecke, heute auch angepasst auf sozialwissenschaftliche Themen.\\
Ablauf:
\begin{itemize}
        \item Ein Problem wird in ``Why''-Fragen umgewandelt
        \item Teilnehmer generieren f"unf antworten auf Basis ihres Wissens und ihrer Erfahrungen
        \item Plausibelste Antwort wird erneut in eine ``Why''-Frage umgewandelt
        \item es werden erneut f"unf Antworten generiert
        \item das Ganze wird $5$-mal wiederholt
\end{itemize}

\noindent Zweck:\\
Teilnehmer werden ermutigt, die Ursache des Problems auf eine Art und Weise zu theoretisieren, die unbegr"undete Annahmen verhindert w"ahrend die Verbindung zu ihren erlebten Erfahrungen aufrecht erhalten bleibt. 

\subsubsection{Die Maplewood Participatory Action Research Gruppe setzte sich nicht bloß aus Akademikerinnen zusammen. Welche anderen Personen waren involviert?}
\begin{itemize}
        \item Eine wei"se Studentin (graduate student) mit Hintergrund in sozialer Schularbeit
        \item Ein wei"ses weibliches Fakult"atsmitglied der  Universit“at mit Hintergrund in Community-/Sozialpsychologie
        \item Vier nicht-graduierte Forschungsassistenten, die alle in einer lokalen "offentlichen Universit"at eingeschrieben sind
        \item Eine fifth-grade-Lehrerin (nahm an allen Sitzungen teil und war an der Unterrichtsplanung beteiligt)
\end{itemize}

\subsubsection{Die Autorinnen berichten ein Cohens Kappa von $80,69\%$. F"ur welche Aussage wird dieses Ma"s herangezogen?}
Cohens Kappa wurde zur Beurteilung der "Ubereinstimmung zweier Kodierer herangezogen. Ein Wert von 80.69 wird als sehr gute "Ubereinstimmung interpretiert.

\subsubsection{Unter ``initial problem definitions'': Warum werden die Aussagen der Sch"ulerinnen nicht als ``second-order''-Aussagen gewertet?}
Genannte L"osungen (f"ur das urspr"ungliche Problem der verschmutzten Schultoiletten) wurden oft mit Ressourcen verbunden, betrafen aber Ver"anderungen erster Ordnung, da zugrundeliegende strukturelle Probleme nicht angetastet werden.\\
Ursachen f"ur Probleme bezogen sich oft auf Schuld, die auf Individuen oder Gruppen geschoben wurde.

\subsubsection{Welche Herausforderungen/Einschr"ankungen beschreiben die Autorinnen, die auftreten, wenn Sch"ulerinnen ihre Sichtweisen auf Probleme innerhalb der Schule beschreiben sollen?}
Ursachen f"ur Probleme bezogen sich oft auf Schuld, die auf Individuen oder Gruppen geschoben wurde. Nicht "uberraschend, dank der dominierenden westlichen soziokulturellen Sichtweise, die junge Menschen als unreif, unverantwortlich und "uberwachungsbed"urftig darstellt. Probleme werden oft als kindliches Verhalten konzeptualisiert.\\
Wenn bspw. j"ungere Sch"uler als die Ursache f"ur Toilettenprobleme benannt werden, werden m"ogliche strukturelle Gr"unde f"ur dieses verhalten missachtet. 

\subsubsection{Welche Konsequenzen haben diese Sichtweisen f"ur die L"osungen, die Sch"ulerinnen f"ur die von ihnen genannten Probleme anbieten?}
Sch"uler entwickeln L"osungen, die der Art der Probleme entsprechen und logisch erscheinen. Wenn ein Individuum oder eine Gruppe als Ursache oder Teil des Problems identifiziert werden konnte, waren L"osungen oft strafend. In F"allen in denen das Problem nicht Individuen zugeordnet werden kann, tendieren die Sch"uler zu materiellen L"osungen.

\subsubsection{Welche Ver"anderung in der Qualit"at der Problembeschreibungen stellen die Autorinnen im Verlauf des vorgestellten ``Five-Whys'' Prozesses (``toilets are unflushed'') fest?}
Die beiden ersten Durchg"ange tendierten zu individuellen Seiten des Problems, was darauf hinweist, dass das Problem in diesem Rahmen individuell konzipiert wurde.  Im sp"ateren Verlauf fokussierten die Sch"uler auf Ressourcen, gingen weg von individuellen Defiziten und brachten systemische institutionelle Probleme in den Vordergrund. Die Daten zeigen, dass die gefundenen Probleme und Ursachen immer struktureller wurden, angefangen vom individuellen Level bis zum systemischen Level des Ressourcenzugangs. 

