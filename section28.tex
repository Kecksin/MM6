\subsection{Fragen und Antworten zum Text von \textcite{shih_costs_2010}}
\subsubsection{Was ist \emph{identity switching}? Inwieweit ist dieser Prozess hilfreich, um mit gruppenbezogener Stigmatisierung umzugehen?}
\emph{Identity switching} meint das Betonen von unterschiedlichen Identit"aten in Abh"angigkeit vom Kontext. In einer Situation, in der das Selbstwertgef"uhl durch eine bestimmte Identit"at bedroht ist, kann durch Wechsel der Identit"at der Selbstwert aufrecht erhalten werden.

\subsubsection{Welche f"orderlichen Konsequenzen k"onnen mit einer multiplen sozialen Identit"at (einem komplexen Selbstkonzept) einhergehen?}
Ein komplexeres Selbstbild\ldots
\begin{itemize}
        \item ist ein Puffer gegen Stress und Depression
        \item f"uhrt zu mehr psychischer Gesundheit (z.B. Selbstwirksamkeit, Selbstwertgef"uhl, Zufriedenheit)
        \item erm"oglicht multiple Quellen der sozialen Unterst"utzung
        \item und hat noch ein paar mehr Konsequenzen, die "ahnlich sind
\end{itemize}

\subsubsection{Worauf bezieht sich \emph{identity adaptiveness}? Erkl"aren Sie anhand eines Beispiels aus dem deutschen Kontext!}
\emph{Identity adaptiveness} meint das Ausma"s, in dem eine Identit"at in einem bestimmten Kontext mit einem positiven Stereotyp verbunden ist.

\subsubsection{Wie kann \emph{identity switching} von \emph{frame switching} abgegrenzt werden?}
\emph{(Cultural) frame switching} bezieht sich auf Ph"anomene und Probleme des interkulturellen Austauschs, w"ahrend \emph{identity switching} sich auch Ver"anderungen intraindividueller Art bezieht. Frame switching ist also eher behavioural, identity switching eher perzpetuell.

\subsubsection{Welche Ziele und Vorteile k"onnen durch \emph{identity switching} erreicht werden? Welche Kosten und Nachteile sind damit verbunden?}

\begin{itemize}
        \item \textbf{Vorteile und Ziele}
                \begin{itemize}
                        \item \emph{Passing:} die adaptivere Identit'at in den Vordergrund bringen um seine Ziele zu erreichen
                        \item Selbstwert aufrechterhalten
                        \item Verhindern von negativen Konsequenzen, die mit einer Identit"at verbunden sind
                \end{itemize}
        \item \textbf{Nachteile und Kosten}
                \begin{itemize}
                        \item Fragmentarisierung des Selbst
                        \item Gesundheitliche Folgen bei chronischer Instabilit"at
                \end{itemize}
\end{itemize}

\subsubsection{Welche Faktoren moderieren die Effekte von \emph{identity switching}?}
\begin{itemize}
        \item Inter-domain vs. Intra-domain switching
        \item Anzahl der Identit"aten
        \item Integration der Identit"aten
\end{itemize}

\subsubsection{Erl"autern Sie anhand von konkreten Beispielen, mit welchen Ma"snahmen auf Meso- und Makroebene \emph{identity adaptiveness} gef"ordert werden kann!}
\begin{itemize}
        \item \textbf{Anzahl der verf"ugbaren Identit"aten erh"ohen} z.B. durch m"ogliche Kategorien im Census-Fragebogen
        \item \textbf{Destigmatisierung von Identit"aten} auf Basis von empirischen Befunden (z.B. zu vermeintlichen Geschlechterunterschieden) oder durch Rollenmodelle 
        \item \textbf{Neutrales ``Framing'' von Aufgaben und Kontexten} durch Vermeidung der Betonung von z.B. vermeintlich typisch weiblichen/m"annlichen Attributen (``dies ist ein visuell-r"aumlicher Test'')
\end{itemize}
