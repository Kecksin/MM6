\subsection{Fragen und Antworten zu \textcite{wang_photovoice:_1997}}
\subsubsection{Welche drei Ziele unterscheiden die Autoren?}
\begin{itemize}
        \item Community Mitglieder in die Lage versetzten, die St"arken und Bed"urfnisse der Community festzuhalten
        \item F"orderung von kritischer Auseindandersetzung mit und Wissen "uber wichtige Themen durch Gruppendiskussionen
        \item Erreichen von Politikern
\end{itemize}

\subsubsection{Auf welchen theoretischen und konzeptuellen "Uberlegungen basiert Photovoice?}
\begin{enumerate}
        \item Theoretische Arbeiten zum Training von Kritischem Denken (Freires Methode), Feminist Theory und Dokumentarfotograhie
        \item Kritik an den Annahmen zu Urheberschaft von Repr"asentationen und Dokumentationen (z.B. m"annlich dominierte Sichtweise)
        \item Erfahrungen aus dem Projekt mit Frauen in Yuhann
\end{enumerate}

\subsubsection{Welche Herausforderungen, bzw. Nachteile diskutieren die Autoren?}
\begin{itemize}
        \item Angst vor dem Ergebnis der Studie k"onnte zu Selbstzensur f"uhren
        \item Pers"onliches Urteil entscheidet, was photografiert wird und was nicht
        \item die ``kleinen Leute'' fotografieren zwar, aber wer verteilt die Mittel? (Gefahr der Reproduktion der Schichtenbildung)
        \item hohe Anforderung an Kapital, Transport und Kommunikation
        \item Wie sollen Bilder ausgewertet werden?
        \item Sind Fotos das beste Mittel? W"aren Erz"ahlungen vielleicht besser?
\end{itemize}

\subsubsection{Welche Einstellungen und F"ahigkeiten sollten Moderatoren besitzen?}
\begin{itemize}
        \item Verpflichtung zu sozialer Ver"anderung
        \item Erkennen der politischen Natur der Bilder
        \item Sensibilit"at gg"u. Aspekten von Macht und Ethik im Zusammenhang mit Kameras
        \item Erkennen von unterschiedlichen Arten des Fotografierens
        \item F"ahigkeit den Umgang mit einer Kamera zu vermitteln
        \item Verst"andnis von Photovoice als ein Prozess aus Diskussion und Handlung
        \item F"ahigkeit, einen Dialog zu den Bildern anzuregen
        \item Wissen "uber lokale Begebenheiten
\end{itemize}

\subsubsection{Was ist mit interner und externer Replikation gemeint}
\emph{Interne} Replikation meint, wenn die gleiche Person ein Thema mehr als einmal identifiziert. \emph{Externe} Replikation meint das Identifizieren eines Themas durch unterschiedliche Personen.
