\subsection{Fragen und Antworten zum Text von \textcite{renault_section_2013}}
\subsubsection{Wof"ur steht das Akronym \emph{SWOT}?}
Strengths, Weaknesses, Opportunities und Threats.

\subsubsection{Warum kann SWOT ein brauchbares Tool im Rahmen von community-psychologischer Pr"avention und Intervention sein?}
Die Methode hat sich als effektiv erwiesen, ist einfach und vielseitig einsetzbar. Sie eignet sich f"ur
\begin{itemize}
        \item Auslotung von neuen M"oglichkeiten
        \item Entscheidungshilfe f"ur das weitere Vorgehen (durch das Identifizieren von guten Gelegenheiten und Gefahren)
        \item Herausfinden, wo Ver"anderung m"oglich ist und welche Priorit"aten es gibt
        \item Anpassung und Verfeinerung von Ma'snahmen w"ahrend des Prozesses
\end{itemize}

\subsubsection{Inwiefern kann aus einem Risiko (\emph{Threat}) eine Chance (\emph{Opportunity}) werden?}
Wurde ein Risiko identifiziert (z.B. Dosen k"onnen nicht recycelt werden), kann eine Ver"anderung neue Chancen erm"oglichen (z.B. recycelbare Dosen "offnen die T"uren f"ur den europ"aischen Raum).

\subsubsection{Was sind zu ber"ucksichtigende Faktoren f"ur die Analyse der St"arken (\emph{Strenghths}) und Schw"achen (\emph{Weaknesses})?}
Allgemein sind das \emph{Ressourcen} (personell, physikalisch, finanziell, Programme, Aktivit"aten) und \emph{Erfahrungen}. 

\subsubsection{Zwar entstammt SWOT dem Wirtschaftsbereich, jedoch k"onnen auch f"ur community-psychologische Pr"aventionen und Interventionen Risiken ausgemacht werden. Welche beispielsweise?}
Hier muss der Begriff des \emph{wettbewerbs} in einem community-psychologischen Sinn umgedeutet werden. Es k"onnte z.B. einen Wettbewerb um die Zeit der Rezipienten von Ma"snahmen geben, oder Ma"snahmen f"ur Pr"avention und Intervention konkurrieren um F"ordergelder.

\subsubsection{Bei der Durchf"uhrung einer SWOT-Analyse lohnt es sich, die Sichtweisen verschiedener Stakeholder einzubeziehen. Wie kann dies stattfinden? Welche Techniken und Methoden werden beschrieben?}
Alle sollten dazu aufgefordert werden, ihre SWOTs zu nennen, am besten aufzuschreiben. Je nachdem, wie wieviele Stakeholder beteiligt sind, lohnt sich eine Gruppenbildung. In den Gruppen kann ein Brainstorming stattfinden. Je mehr Ideen notiert werden, desto eher sind auch gute dabei. Am Ende m"ussen die Gruppen zusammenkommen. Die Ergebnisse k"onnen entweder gruppenweise oder nach SWOT-Kategorien zusammengef"uhrt werden.

\subsubsection{Welche F"ahigkeiten sollte die Person haben, die die SWOT-Analyse leitet?}
\begin{itemize}
        \item zuh"oren 
        \item Gruppen leiten 
        \item Prozess am laufen halten 
        \item Diskussion bei der Sache halten 
\end{itemize}
