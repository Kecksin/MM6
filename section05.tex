\subsection{Fragen und Antworten zum Text von \textcite{rhew_sustained_2013}}
\subsubsection{Was beinhaltet das Programm Communities That Care?}


CTC ist ein System, das Communities in die Lage versetzen soll, Ma"snahmen zu ergreifen, die Risiko- und Schutzfaktoren adressieren. Dabei geht es um Drogenmissbrauch und antisoziale Verhaltensweisen bei Jugendlichen. Das geschieht durch die angemessene Auswahl, Implementierung und "Uberwachung dieser Ma"ssnahmen. CTC besteht aus $5$ Phasen.

\begin{figure}[<+htpb+>]
        \begin{center}
                \begin{tikzpicture}[mindmap,font=\large\scshape,grow cyclic, every node/.style={concept, execute at begin node=\hskip0pt}, concept color=black!20,
                        %child concept/.append style={color=gray,fill=white},
                        level 1/.append style ={level distance=4cm,sibling distance=1cm,sibling angle=72}
                        ]
                        \node{Communities That Care}
                        child{node{Wissenschaftsbasierter Ansatz}}
                        child{node{Bereitschaft der Community}}
                        child{node{Normen gegen Drogenmissbrauch}}
                        child{node{Kollaboration von Organisationen}}
                        child{node{Nutzung einer sozialen Entwicklungsstrategie}};
                \end{tikzpicture}
        \end{center}
        \caption{Die $5$ Phasen von CTC nach \textcite{rhew_sustained_2013}.}
        \label{fig:fuenfphasen}
\end{figure}

\subsubsection{Was war das Ziel der Studie von \textcite{rhew_sustained_2013}?}

Es sollte gepr"uft werden, ob die Ma"snahmen auch $1.5$ Jahre nach Beendigung der F"orderzahlungen weiterhin wirken.

\subsubsection{Beschreiben sie die methodische Vorgehensweise}

Zu insgesamt $4$ Zeitpunkten wurden \emph{Schl"usselfiguren} mit dem \emph{Key Informant Interview} befragt. Dabei wurden zun"achst eine Reihe von Community Leaders ausgew"ahlt, die dann wiederum andere Personen vorschlagen sollten. Erfasst wurden die $5$ Elemente von CTC plus ein zus"atzliches Element, sowie demographische Variablen.

\begin{description}
        \item[Wissenschaftsbasierter Ansatz] Die Schl"usselfiguren wurden nach ihrem Wissen zu wissenschaftlichen Pr"aventionskonzepten, dem Gebrauch von epidemiologischen Daten, der Nutzung von empirisch fundierten Ma"snahmen und nach Monitoringaktivit"aten befragt. Das alles mit $21$ Items.
        \item[Bereitschaft der Community] Dies wurde als eine latente Variable zweiten Grades konzipiert, die wiederum mit zwei weiteren latenten Variablen gemessen wurde: \emph{Bereitschaft der Community Mitglieder} und \emph{Bereitschaft der Community Leader}. Gefragt wurde, inwiefern die Community Mitglieder (bzw. die Leader) an pr"aventive Ma"snahmen glauben, "uber deren Einsatz in der Community bescheid wissen und bereit w"aren, daf"ur h"ohere Steuern zu bezahlen. 
        \item[Erw"unschte F"orderh"ohe] Erfragt mit einem Item: ``Wenn sie entscheiden k"onnten, wieviel Geld w"urden sie f"ur juristische Ma"snahmen, Behandlung und Pr"avention ausgeben?''
        \item[Normen gegen Dogenmissbrauch] Erfragt als latente Variable mittels $6$ Items.
        \item[Kollaboration bei der Pr"avention] Konzipiert als latente Variable zweiten Grades mittels zwei latenten Variablen: \emph{sektoriale Kollaboration} und \emph{pr"aventive Kollaboration}. Sektoriale Kollaboration wurde erfasst "uber die H"aufigkeit des Kontakts mit anderen Sektoren, pr"aventive Kollaboration wurde erfasst "uber 9 Items zu verschiedenen Aktivit"aten.
        \item[Nutzung einer sozialen Entwicklungsstrategie] Konzipiert als latente Variable mit $5$ Items. Unter anderem geht es hier um die M"oglichkeiten f"ur Jugendliche, prosozialen Aktivit"aten nachzugehen, neue Fertigkeiten zu lernen. Auch, inwiefern sie gelobt werden. 
\end{description}

\subsubsection{Wie ist der langfristige Effekt von CTC zu beurteilen?}
Durchaus positiv. Auch $1.5$ Jahre nach Ende der Projektfinzanzierung\ldots
\begin{itemize}
        \item wurden mehr wissenschaftlich fundierte Programme verwendet
        \item wurden mehr F"ordergelder f"ur pr"aventive Ma"snahmen bewilligt
        \item sind Normen gegen Drogenmissbrauch gestiegen
\end{itemize}

\noindent Keine Unterschiede zwischen CTC- und Kontrollgruppen gab es bei folgenden Punkten:
\begin{itemize}
        \item generelle Bereitschaft der Community
        \item Kollaboration von Organisationen innerhalb der Community
        \item Nutzung einer sozialen Entwicklungsstrategie
\end{itemize}

\subsubsection{Was sind die Schw"achen der Studie?}
Zum einen wurden nur kleine bis mittelgro"se St"adte untersucht. Die Frage ist also, inwiefern die Ergebnisse generalisierbar sind. Ausserdem k"onnte es sein, dass die befragten Community Leaders in sozial erw"unschter Weise geantweortet habt, also besch"onigend. Allerdings gibt es keinen generellen ``Halo''-Effekt.

\subsubsection{Welche Empfehlungen f"ur die Entwicklung und Implenetation von "ahnlichen Programmen lassen sich ableiten?}
F"ordergelder in Zusammenhang mit Unterst"utzung bei der Implementation von Pr"aventionsm"snahmen scheinen geeingnet zu sein, langfristige Wirkungen zu entfalten. Zentraler Punkt ist die Verwendung von wissenschaftlich fundierten Methoden und Programmen.
