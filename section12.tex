\subsection{Fragen zum Text von \textcite{catalani_photovoice:_2010}}
\subsubsection{Welche drei Fragen werden in dem Review adressiert?}
\begin{itemize}
        \item Wie ist die Photovoice-Methode definiert?
        \item Welche Ergebnisse sind mit Photovoice assoziiert?
        \item Wie h"angt das Ausma"s an Partizipation mit Prozessen und Ergebnissen zusammen?
\end{itemize}

\subsubsection{F"ur welche Forschungsfragen, in welchen Kontexten und mit welchen Zielgruppen wurde Photovoice eingesetzt?}
Forschungsfragen: vor allem im Gesundheitswesen und in Fragen der sozialen Gerechtigkeit. Anwendungsf"alle waren z.B.:
\begin{itemize}
        \item Epidemien bei Infektionskrankheiten
        \item Chronische Krankheiten
        \item Politische Gewalt
        \item Diskriminierung
\end{itemize}

Es gab die unterschiedlichsten Zielgruppen:
\begin{itemize}
        \item Jugendliche
        \item Senioren
        \item Amerikaner
        \item Europ"aer
        \item Asiaten
\end{itemize}

\subsubsection{Skizzieren sie das Vorgehen bei der Auswahl der Prim"arstudien. Welche Kriterien wurden angelegt? W"urden sie anders vorgehen? Wenn ja, wie und wieso?}
Nur peer-reviewed Studien zu gesundheitlichen Themen wurden einbezogen. Buchkapitel oder Abschlussarbeiten wurden nicht ber"ucksichtigt.

\subsubsection{Skizzieren sie die Auswertungsstrategie. Welche Vor- und Nachteile hat diese? W"urden sie anders vorgehen? Wenn ja, wie und wieso?}
Alle Texte wurden in Bezug auf mehrere Kategorien kodiert. Zus"atzlich dazu wurden Indikatoren gebildet, die auf besondere Qualit"at der Forschung im Bezug auf \emph{Photovoice}, \emph{Koalitionsbildung} und \emph{Nachhaltigkeit} hinweisen.

Das Ausma"s an Partizipation wurde mit Viswanathan et al.'s (2010) \emph{quality of participation tool} bestimmt. Drei Kategorien wurden vergeben: niedrige, mittlere und hohe Partizpation.

\subsubsection{Welche Einschr"ankungen der Forschungssynthese werden dikutiert?}
\begin{itemize}
        \item nur peer-reviewed Studien 
        \item nur gesundheitsbezogene Artikel
        \item Publication Bias
        \item nur ein Author hat alle Studien kodiert
\end{itemize}

\subsubsection{Wof"ur steht das Akronym SHOWeD? Wann und wof"ur kann es eingesetzt werden?}
SHOWed steht f"ur:
\begin{itemize}
        \item What do you \emph{See} here?
        \item What ist \emph{Happening} here?
        \item How does it relate to \emph{Our} lives?
        \item \emph{Why} does this problem, strength or concern \emph{exist}?
        \item What can we \emph{Do} about it?
\end{itemize}

\subsubsection{Welcher Zusammenhang zwischen dem Ausma"s an Partizipation und der Anzahl an Teilnehmenden wird berichtet?}
Keiner.

\subsubsection{Welcher Zusammenhang zwischen dem Ausma"s an Partizipation und der Dauer der Projekte wird berichtet?}
Je l"anger die Projekte, desto mehr Partizipation.

\subsubsection{Welche Zusammenh"ange zwischen dem Ausma"s an Partizipation und den Eigenschaften der Teilnehmenden werden berichtet?}
Keine.

\subsubsection{Welche Argumente sprechen f"ur und welche gegen ein formales Training der Photovoice Methode?}
Contra:
\begin{itemize}
        \item Die Fotografiestil wird als Datenquelle f"ur soziale Konstruktionen betrachtet
        \item die Teilnehmenden sollen m"oglichst wenig beeinflusst werden (naturalistischer Stil)
\end{itemize}

\subsubsection{Welche Art von Forschungsfragen k"onnen mit Photovoice adressiert werden? Welche Art von Daten liefert die Photovoice Methode?}
Photovoice liefert:
\begin{itemize}
        \item Disskussionstranskriptionen
        \item Interviewtranskriptionen
        \item Fotografien
\end{itemize}

\subsubsection{Welche Funktion haben die Gruppendiskussionen, in denen die Fotos besprochen werden?}
Sie sind der eigentliche Gegenstand der Analysen.

\subsubsection{Welche drei Ergebnisklassen werden definiert? Welche Zusammenh"ange zwischen dem Ausma"s an Partizipation und der jeweiligen Ergebnisklasse werden berichtet?}
\begin{itemize}
        \item mehr Engagement f"ur Handlungen und Interessensvertretung
        \item besseres Verst"andnis von Bed"urfnissen und St"arken
        \item Zugewinn von Empowerment
\end{itemize}

Je mehr Partizipation, desto mehr Handlungen, desto mehr Empowerment und desto mehr Verst"andnis von Bed"urfnissen und St"arken.

\subsubsection{Welche Schw"achen des bisherigen Forschungsstandes werden identifiziert? Was sind offene Fragen f"ur die Zukunft?}
Schw"achen, die zuk"unftig angegangen werden sollen:
\begin{itemize}
        \item die Beschreibung der Methoden von Photovoice ist ungenau
        \item es ist unklar, wie das Ausma"s an Partizipation berichtet werden soll
        \item obwohl Photovoice dazu dienen soll, auf Community Ebene Wirkung zu zeigen, wird diese Ebene in den Analysen nicht ber"ucksichtigt
\end{itemize}
