\subsection{Fragen und Antworten zum Text von \textcite{steffens_diskriminierung_2009}}
\subsubsection{Wie kann es sich auswirken, wenn Schwule, Lesben und Bisexuelle aus Angst ihre sexuelle Orientierung verbergen?}
\begin{itemize}
        \item niedriges Wohlbefinden
        \item Gesundheitsbeeintr"achtigungen
        \item Aufrechterhaltung von gesellschaftlichen Ressentiments
        \item Vermeidung von sozialen Kontakten
        \item Ver"argen von Sozialkontakten durch reservierte Haltung
\end{itemize}

\subsubsection{Welche Gesetze gab und gibt es, die zur Diskriminierung oder Gleichstellung beitragen? welche Schritte sind noch notwendig f"ur die tats"achliche Gleichstellung?}
Gesetze:
\begin{itemize}
        \item \S 175 StGB (BRD) stellte sexuelle Handlungen zwischen M"annern unter Strafe (1872 -- 1994, BRD) 
        \item \S 151 stellte homosexuelle Handlungen mit Jugendlichen f"ur M"anner und Frauen unter Strafe (1964 -- 1988, DDR)
        \item Gesetz "uber die eingetragene Lebenspartnerschaft (2001)
        \item AGG (2006)
\end{itemize}

\noindent Notwendig f"ur die tats"achliche Gleichstellung:
\begin{itemize}
        \item Einordnung des Lebenspartnerschaftsrechts in das Personenstandsgesetz ("Offnung der Standes"amter)
        \item Gleichstellung verpartnerter Beamten/innen bei Familienzuschlag und Pension
        \item Gleichstellung der Lebenspartner/innen mit Ehegatten im Steuerrecht
        \item Adoptionsrecht
\end{itemize}

\subsubsection{Was sind Beispiele f"ur strukturelle Diskriminierung?}
\begin{itemize}
        \item Versicherungen
                \begin{itemize}
                        \item keine Lebensversicherungen bei gleichgeschlechtlichen Beg"unstigten
                        \item pauschale Verweisung auf erh"ohtes AIDS-Risiko
                        \item keine g"unstigen Partnertarife
                \end{itemize}
        \item Verweigerung von Bedienung im Gast"attenbereich
        \item Wohnungssuche
\end{itemize}

\subsubsection{Wie hoch ist die Pr"avalenz von allt"aglichen Diskriminierungserfahrungen in Deutschland und "Osterreich?}
Deutschland:
\begin{itemize}
        \item Frauen: Beschimpfungen 36\%, Ausgrenzung 28\%, Aufforderungen zu sexuellen Handlungen 16\%, gezwungen zu sexuellen Handlungen 8\%
        \item M"annern: Beschimpfungen 39\%, bedr"angt 14\%, bespuckt 4\%, beworfen 5\%, Eigentum besch"adigt 8\%, bestohlen 9\%, beraubt 4\%, k"orperlicher Angriff (nicht verletzt) 10\%, k"orperlicher Angriff (leicht verletzt) 5\%, k"orperlicher Angriff (schwer verletzt) 1\%
\end{itemize}

\noindent "Osterreich:
\begin{itemize}
        \item Frauen: 13\% verbale Diskriminierung, 13\% k"orperliche Gewalt
        \item M"anner: 21\% verbale Diskriminierung, 24\% k"operliche Gewalt
\end{itemize}

\subsubsection{Warum werden schwule M"anner st"arker diskriminiert als bisexuelle Frauen?}
\begin{itemize}
        \item M"anner "uben eher Gewalt aus, und das auch eher gegen M"anner
        \item Heterosexuelle finden Schwule schlimmer als Lesben
        \item Bisezuelle werden evtl. eher als heterosexuell wahrgenommen, Schwule dagegen sind leichter als solche zu erkennen
\end{itemize}

\subsubsection{Warum werden Konversionstherapien heute nicht mehr empfohlen?}
\begin{itemize}
        \item keine empirischen Befunde f"ur gesundheitsf"ordernde Effekte
        \item negative Effekte wie depressive oder "angstliche Symptome
\end{itemize}

\subsubsection{Was kritisieren \textcite{steffens_diskriminierung_2009} am ICD-10?}
Es gibt immer noch drei Kategorien zur sexuellen Orientierung (die in der Fachwelt als revisionsbed"urftig angesehen werden).

\subsubsection{Worauf sollten Therapeutinnen in der Arbeit mit Schwulen und Lesben achten?}
\begin{itemize}
        \item sexuelle Orientierung ist nicht die Ursache eines Problems
        \item Bedeutung der sexuellen Orientierung nicht untersch"atzen
\end{itemize}

\subsubsection{Diskriminierung im Zusammenhang mit sexueller Orientierung, Angst vor Diskriminierung und negative Einstellungen zur eigenen sexuellen Orientierung wirken sich negativ auf die psychische Gesundheit aus. Welche Befunde st"utzen diese Aussagen?}
\begin{itemize}
        \item Mak, Poon, Pun und Cheung (2007) fanden in einer Metanalyse signifikante Beziehungen zwischen Stigmatisierung und allen 3 Kriterien
        \item Pl"oderl, Trembley und Fartacek (2007) fanden eine erh"ohte Pr"avalenz von Suizidversuchen bei (verbaler oder k"operlicher) Diskriminierung
        \item De Graf, Sandford und ten Have (2006) fanden ein erh"ohtes Suizidrisiko von homosexuellen gg"u. heterosexuellen M"annern in einer niederl"andischen repr"asentativen Stichprobe
        \item Mays und Cochran (2001) fanden erh"ohte Pr"avalenzen f"ur wahrgenommene Diskriminierung und psychische St"orungen in einer amerikanischen repr"asentativen Stichprobe von Schwulen, Lesben und Bisexuellen gg"u. Heterosexuellen
\end{itemize}
