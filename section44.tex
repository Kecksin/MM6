\section{Evaluation}
\subsection{Fragen und Antworten zum Text von \textcite{cullen_forms_2011}}
\subsubsection{Was ist Partizipative Evaluation? Warum ist es so schwer, eine Definition zu finden?}
Man muss wohl sagen, dass es sich um einen Oberbegriff handelt. Vermutlich ist eine einheitliche Definition schwer, weil der Begriff \emph{Partizipation} unklar ist. Dass es eine \emph{Teilnahme} gibt ist zwar klar, aber wer genau und wie genau partizipiert wird, dar"uber gibt es unterschiedliche Meinungen.

\subsubsection{Was sind Ziele der Partizipativen Evaluation?}
Es gibt $3$ Arten von Zielen:
\begin{itemize}
        \item pragmatische (Entscheidungshilfe)
        \item politische 
        \item epistemologische
\end{itemize}

\subsubsection{Wie lauten die $3$ Dimensionen zur Unterscheidung von Partizipativen Evaluationen nach Cousins \& Whitmore $(1998)$?}
\begin{itemize}
        \item Kontrolle
        \item Auswahl der beteiligten Stakeholder
        \item Tiefe (inwieweit sind die Stakeholder beteiligt)
\end{itemize}

\subsubsection{Was sind die drei Grunds"atze des \emph{Stakeholder-Based Model}?}
\begin{itemize}
        \item Nutzung der Evaluationsergebnisse steigern
        \item Bandbreite an beteiligten Stakeholdern erh"ohen
        \item mehr Kontrolle f"ur Stakeholder
\end{itemize}

\subsubsection{Was sind zentrale Unterscheidungspunkte von \emph{Transformative Participatory Evaluation (T-PE)} und \emph{Practical Participatory Evaluation (P-PE)}?}
In beiden Ans"atzen sind die Stakeholder an allen Phasen der Evaluation beteiligt. Bei P-PE soll die Steigerung der Nutzung der Ergebnisse durch Steigerung von \emph{Ownership} herbeigef"uhrt werden. Dazu werden vor allem \emph{Manager, Sponsoren, etc.} eingebunden. Bei der T-PE dagegen geht es um Empowerment und soziale Ver"anderung. Demnach sind vor allem \emph{benachteiligte und unterdr"uckte Gruppen} beteiligt.

\subsubsection{Welches sind die unterscheidenden Charakteristiken der \emph{Responsive Evaluation}?}
Das entscheidende ist die Responsivit"at f"ur sich herauskristallisierende Punkte.

\subsubsection{Was macht die \emph{Utilized Focussed Evaluation} aus?}
Damit die Ergebnisse von Nutzen sind, m"ussen sie f"ur die Beteiligten relevant sein. Deshalb sollen nur diejenigen an der Evaluation partizipieren, die auch im zu evaluierenden Programm partizipieren.

\subsubsection{In welchen Punkten unterscheiden sich \emph{Rapid Rural Appraisal (RRA)} und \emph{Participatory Rural Appraisal (PRA) voneinander}?}
\begin{itemize}
        \item \emph{RRA}: Wie k"onnen Aussenstehende m"oglichst kosteng"unstig Daten in l"andlichen Gebieten sammeln? 
        \item \emph{PRA}: Wie k"onnen Stakeholder m"oglichst kosteng"unstig ihre Daten sammeln? 
\end{itemize}

\subsubsection{Beim zusammenfassenden Vergleich der Ans"atze: Was sind die auff"alligsten Ergebnisse im Hinblick auf die Dimensionen von Cousins \& Whitmore $1998)$?}
Es gibt eine gro"se Variabilit"at. Das unterstreicht das Problem, dass es keine einheitliche Definition von Partizipativer Evaluation gibt.


