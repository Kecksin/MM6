\subsection{Fragen und Antworten zum Text von \textcite{tropp_summary_2011}}

\subsubsection{F"ur welche Faktoren von Intergruppenkontakt berichtet \textcite{tropp_summary_2011}Ergebnisse? Was sind die zentralen Befunde und Schlussfolgerungen, die aus der Metaanalyse gezogen werden k"onnen? }
Faktoren von Intergruppenkontakt:
\begin{itemize}
        \item St"arke/Intensit"at des Kontakts
        \item oberfl"achlicher vs. freundschaftlicher Kontakt 
        \item optimale Kontaktbedingungen 
\end{itemize}

\noindent Befunde:
\begin{itemize}
        \item st"arkerer Intergruppenkontakt >> geringere Intergruppenvorurteile
        \item Effekte von Kontakt k"onnen von Erfahrungen auf der individuellen Ebene auf die Ebene der Gruppe als Ganzes generalisiert werden 
        \item Freundschaftskontakt hat eine st"arkere Reduktion von Vorurteilen zur Folge 
        \item positive Ergebnisse von Kontakt besonders wahrscheinlich, wenn optimale Bedingungen herrschen: Gleichheit \& Kooperation >> signifikant st"arkere Reduktion von Vorurteilen bei Kindern und Jugendlichen im schulischen Bereich 
        \item Kontakt-Studien, die strengere Forschungsmethoden verwenden, zeigen eher eine Reduktion von Vorurteilen durch Kontakt 
        \item ``extended effects of contact'': positive Einstellungen und Kontaktbereitschaft von Personen ohne Kontakt zu anderen Gruppe kann dadurch gef"ordert werden, wenn diese Personen wissen, dass andere Mitglieder ihrer Gruppe mit Mitgliedern anderer Gruppen befreundet sind
        \item positive Effekte von Kontakt unter Mitgliedern der Minorit"at weniger ausgepr"agt als unter Mitgliedern der Majorit"at 
        \item geringe Vorurteile vorherzusagen ist besonders wahrscheinlich, wenn die affektive Dimension von Vourteilen beurteilt wird, Effekte scheinen bei kognitiven Dimensionen (z.B. Stereotype) eher schw"acher zu sein
        \item Mediatoren von Kontakteffekten: Angst, Wissen und Empathie 
\end{itemize}

\subsubsection{ Vergleichen Sie die Pr"asentation der Ergebnisse des Textes mit der Darstellung in einem Forschungsartikel. Was f"allt Ihnen auf? }
Der Artikel ist eher wie eine Pr"asentation aufgebaut. Scheint nicht an ein Fachpublikum gerichtet zu sein. Au"ser den Korrelationen werden keine statistischen Ma�e angegeben. Ist nicht wie ein typischer Forschungsartikel aufgebaut: Einleitung/Methoden/Ergebnisse/Diskussion/zuk"unftige Forschung
