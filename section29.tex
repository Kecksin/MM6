\subsection{Fragen und Antworten zum Text von \textcite{luthar_children_2005}}
\subsubsection{Welche Kohorten wurden untersucht und was waren die Ergebnisse?}

F"ur die Zusammensetzung der Gruppen siehe Tabelle \ref{tab:cohorts}
\begin{table}
        \centering
        \begin{tabular}{c | l | c}
                Name & Klassenstufen & Vergleichsgruppe\\
                \hline
                \hline
                Kohorte 1 & $10$ -- Universit"at & Stadtkinder\\
                Kohorte 2 & $6$ -- $7$ & keine \\
                Kohorte 3 & $6$ -- $11$ & Stadtkinder\\
                \hline
        \end{tabular}
        \caption{Kohorten aus \textcite{luthar_children_2005}.}
        \label{tab:cohorts}
\end{table}

In Kohorte 1 gab es mehr Substanzmissbrauch (Alkohol, Zigaretten, Marihuana) und h"ohere Werte in ""Angstlichkeit und Depression bei wohlhabenden Kindern. Depressionswerte waren f"ur M"adchen h"oher als f"ur Jungen. In den anderen zwei Kohorten wurden die Ergebnisse best"atigt, allerdings treten sie erst ab der $7.$ Klasse auf.

\subsubsection{Leistungsdruck und Isolierung k"onnen bei Kugendlichen mit h"oherem sozio"okonomischen Status zu Problemen f"uhren. Was ist damit gemeint?}
H"ohere Depressionswerte und mehr Substanzmissbrauch f"ur perfektionistische Kinder, Kinder deren Eltern das schulische Abschneiden "uberproportional bewerteten und bei Kindern, die wenig Zeit mit ihren Eltern verbracht haben.

\subsubsection{Die Jugendlichen wurden zu sieben verschiedenen Aspekten des Familienlebens befragt. Welche Unterschiede und Gemeinsamkeiten zwischen den Gruppen zeigten sich?}
Siehe Tabelle \ref{tab:familyaspects}
\begin{table}[hb!]
        \centering
        \begin{tabular}{c | c}
                \hline
                Aspekt & Ergebnis\\
                \hline
                gef"uhlte N"ahe zur Mutter & kein Unterschied\\
                gef"uhlte N"ahe zum Vater & kein Unterschied\\
                Betonung von Integrit"at durch Eltern & kein Unterschied\\
                Regelm"a"siges Abendessen mit Eltern & kein Unterschied\\
                \hline
                Kritik durch Eltern & wohlhabende Kinder besser\\
                Supervision nach der Schule & wohlhabende Kinder besser\\
                \hline
                Erwartungen der Eltern & wohlhabende Kinder schlechter\\
                \hline
        \end{tabular}
        \caption{Ergebnisse der Befragung zu Aspekten des Familienlebens \parencite{luthar_children_2005}.}
        \label{tab:familyaspects}
\end{table}

\subsubsection{Womit h"angt Beliebtheit bei Gleichaltrigen in den Subgruppen zusammen?}
Grunds"atzlich h"angt in beiden Extremgruppen Beliebtheit mit Auflehnung gegen die Obrigkeit zusammen. F"ur genaueres siehe Tabelle \ref{tab:peerstatus}.
\begin{table}
        \centering
        \begin{tabular}{c | c | r}
                \hline
                Gruppe & Geschlecht & Merkmal\\
                \hline
                \hline
                wohlhabende Kinder & M"adchen & geringe schulische Anstrengung\\
                & & Nichtbefolgen von Regeln\\
                & & Aggressivit"at\\
                & Jungen & Substanzmissbrauch\\
                \hline
                Stadtkinder & beide & Aggressivit"at\\
                & & Substanzmissbrauch\\
                \hline
        \end{tabular}
        \caption{Zusammenh"ange von Beliebtheit mit diversen Merkmalen \parencite{luthar_children_2005}.}
        \label{tab:peerstatus}
\end{table}

\subsubsection{Was tr"agt dazu bei, dass Jugendliche aus der Oberschicht oft keine Hilfe bekommen, wenn sie depressiv sind oder andere psychische Probleme haben?}
Schulspychologen und "Arzte tun Probleme von wohlhabenden Kindern oft ab und bewerten sie anders als bei sozial schwachen Kindern. 

\subsubsection{Wie kann es sich auswirken, wenn die psychische Gesundheit von reicheren Jugendlichen vernachl"assigt wird?}
Kinder aus wohlhabenden Familien werden selbst mit hoher Wahrscheinlichkeit zu einflussreichen Pers"onlichkeiten, deren Produktivit"at dann durch eine eventuelle Depression beeintr"achtigt ist. Au"serdem neigen psychisch instabile und ungl"uckliche Personen eher dazu, Ressourcen anzuh"aufen, anstatt sie der Gemeinschaft zur"uckzugeben. (Manche sprechen in diesem Zusammenhang gar von Habgier.)
