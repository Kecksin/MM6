\section{Identit"at}
\subsection{Fragen und Antworten zum Text von \textcite{amiot_facilitating_2010}}
\subsubsection{Skiziieren sie die unterschiedlichen Phasen des \emph{model of social identity development and integration} \parencite{amiot_facilitating_2010}}
\begin{itemize}
        \item \textbf{Antizipatorische Kategorisierung.} Verankerung der eigenen Identit"at im Vergleich zur neuen Gruppe, bevor ein erster Kontakt stattgefunden hat
        \item \textbf{Kategorisierung.} Konfrontation mit der fremden Gruppe und Realisierung der Unterschiede. Unterschiedliche Gruppenzugeh"origkeit wird salient.
        \item \textbf{Kompartmentalisierung.} Schrittweise "Ubernahme der neuen Identit"at. Alte und neue Identit"at werden aber noch nicht vermischt sondern existieren nebeneinander in distinkten ``Kompartments''.
        \item \textbf{Integration.} Konkrete Anstrengungen um die Konflikte zwischen Identit"aten zu "uberwinden und Gemeinsamkeiten zu finden. Erleben von \emph{Konsistenz} und \emph{Koh"arenz} bez"uglich der eigenen Identit"at.
\end{itemize}

\subsubsection{Grenzen sie dieses Modell von \emph{Social Identity Complexity Theory}, bzw. von \emph{Social Identity/Self Categorization Theory} ab!}

Das \emph{Social Identity Complexity Model} \parencite{brewer_social_2010} beschreibt unterschiedliche Grade der Komplexit"at von Identit"at. Es werden vier Arten genannt, auf die multiple Gruppenzugeh"origkeiten zu einer komplexeren Identi"at kombiniert werden k"onnen. Es wird postuliert, dass diese vier Grade der Komplexit"at auf einem Kontinuum von wenig komplex bis sehr komplex liegen. (Siehe Fig. \ref{fig:amiot1})

\begin{figure}[hb!]
        \begin{center}
                \begin{tikzpicture}
                        [ex/.style={rectangle, draw, text width=15em, text centered, rounded corners, font=\footnotesize},
                        konz/.style={rectangle, draw, text centered, rounded corners}]
                        %Namen der unterschiedlichen Auspraegungen von Komplexitaet
                        \node [konz,blue](inters)     at(0,0)                {Intersection};
                        \node [konz,magenta](domin)     [right=of inters]       {Dominance};
                        \node [konz,green](compar)    [right=of domin]        {Compartmentalization};
                        \node [konz,red](merge)     [right= of compar]      {Merger};
                        \draw [->] (-1, -1) -- (11, -1);
                        \node [font=\tiny] at (-2,-1) {wenig komplex};
                        \node [font=\tiny] at (12,-1) {komplex};
                        %Erlaeuterungen zu den Auspraegungen
                        \node [ex,blue,below= 1cm of inters] (intersex)     {Vereiningung/Schnittmenge aller Identit"aten ist konstituierend f"ur die Identit"at.};
                        \node [ex,magenta,below= 2.3cm of domin] (dominex)   {Es gibt eine prim"are Gruppenzugeh"origkeit. Alle anderen werden untergeordnet.};
                        \node [ex,green,below= 3.5cm of compar] (comparex)   {Multiple Gruppenidentit"aten existieren nebeneinander und werden kontextabh"angig vorgeholt.};
                        \node [ex,red,below= 1cm of merge] (mergeex)    {Differenzierung zwischen Identit"aten bei gleichzeitiger Integration in eine "ubergeordnete Identit"at.};
                \end{tikzpicture}
        \end{center}
        \caption{Kontinuum der Komplexit"at von Identit"at nach dem \emph{Social Identity Complexity Model} \parencite{brewer_social_2010}}
        \label{fig:amiot1}
\end{figure}

Die \emph{Selbstkategorisierungstheorie} zielt im Unterschied zum Entwicklungsmodell von \textcite{amiot_facilitating_2010} auf kurzfristige Kategorisierungsprozesse ab.

\subsubsection{Was ist definierend f"ur die vierte Phase der \emph{Integration} und was unterscheidet diese Phase von der \emph{Compartmentalization}?}
Bei der Integration werden alle multiplem Gruppenzugeh"origkeiten zu einer neuen Identit"at integriert. Dadurch entsteht das Gef"uhl von \emph{Konsistenz, Koh"arenz} und \emph{Authentizit"at}.

\subsubsection{Welche sozialen Faktoren erleichten oder behindern die Integration von neuen Identit"aten? Welche Ansatzpunkte ergeben sich daraus?}
Die zwei Faktoren sind:
\begin{itemize}
        \item Bedrohung
        \item Soziale Unterst"utzung
\end{itemize}

\noindent Die Autoren nennen f"unf Interventionsans"atze:
\begin{itemize}
        \item Anerkennung der Existenz von Diskriminierung und Rassismus
        \item Verst"andnis, dass Bedrohung die Wurzel aller Spannung zwischen Gruppen ist
        \item Entwicklung einer Identit"at, die alle multiplen Gruppenzugeh"origkeiten umfasst (Integration)
        \item Sicherstellung von sozialer Unterst"utzung f"ur Immigranten
        \item Sicherstellung, dass unterschiedliche kulturelle Identit"aten gleicherma"sen wertgesch"atzt werden
\end{itemize}


