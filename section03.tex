\subsection{Fragen und Antworten zum Text von \textcite{beelmann_scientific_2012}}
\subsubsection{Was ist mit Developmental Crime Prevention (DCP) gemeint?}
DCP bezeichnet pr"aventive Ma"snahmen zur Verhinderung von kriminellen Karrieren durch Reduktion von Risikofaktoren und F"orderung von positiver Entwicklung, basierend auf empirischen Erkenntnissen zu normaler und abweichender Entwicklung von der Geburt bis ins Erwachsenenalter.

\subsubsection{Welche grundlegenden Prinzipien sollten bei der Entwicklung eines Pr"aventionsprogramms beachtet werden?}
Zwei Fragen m"ussen beantwortet werden:
\begin{enumerate}
        \item Wieviel theoretische Fundierung und empirische Evidenz muss vorliegen?
        \item Welche Bedingungen an oben genanntes m"ussen erf"ullt sein, damit man von einer wissenschaftlich fundierten Ma"snahme sprechen kann?
\end{enumerate}

\subsubsection{Wann gilt eine Pr"aventionsma"snahme als evidenzbasiert?}
Wenn die \emph{Efficacy} der Ma"snahme gezeigt wurde.

\subsubsection{Beschreiben sie Beelmanns Modell!}
In Beelmanns Modell gibt es zum einen vier Schritte, die bei der \emph{Entwicklung} einer Ma"snahme beachtet werden m"ussen. Anschlie"send muss das Programm dann nat"urlich auch noch evaluiert werden. Diese \emph{Evaluation} enth"alt drei Elemente. Siehe dazu Fig. \ref{fig:beelmann1}.

\begin{figure}[h!]
        \begin{center}
                \begin{tikzpicture}[vertikal/.style={shape=rectangle,draw,rounded corners,text centered,text width=3.5cm, minimum size=3cm},
                        rund/.style={shape=ellipse,draw},
                        horizontal/.style={shape=rectangle,draw,rounded corners,text centered,text width=2cm}]
                        \node [rund] at (0,0) {Wissenschaftlich Fundierte Pr"aventionsma"snahme};
                        \node [vertikal] at (-6,3) {\textbf{Generelle Legitimation}\\ Gibt es "uberhaupt ein Problem (Pr"avalenz) und wie relevant ist es (Folgen)? };
                        \node [vertikal] at (-2,3) {\textbf{Entwicklungs\-bezo\-gene Fundierung}\\ Befunde zur Entwicklung der f"ur das Programm relevanten Faktoren};
                        \node [vertikal] at (+2,3) {\textbf{Programm Theorie}\\ Was sind Risiko- und protektive Faktoren? Gibt es entwicklungsbezogene Modelle?};
                        \node [vertikal] at (+6,3) {\textbf{Interventionstheorie}\\ Welches Setting, Methode, Intensit"at, etc. ist angemessen f"ur die Zielgruppe?};
                        \node [horizontal] at (-4,-1) {Effektivit"at};
                        \node [horizontal] at (+0,-1) {Efficacy};
                        \node [horizontal] at (+4,-1) {Verbreitung};
                \end{tikzpicture}
        \end{center}
        \caption{Das Modell von Beelmann zur Entwicklung von wissenschaftlich fundierten Pr"aventionsma"snahmen. In den oberen vier K"asten stehen die schritte, die bei der Entwicklung des Programms beachtet werden m"ussen. Im unteren Kasten stehen die Schritte zur Evaluation eines Programms.}
        \label{fig:beelmann1}
\end{figure}

\subsubsection{Was sind Vor- und Nachteile von allgemeinen und gezielten Pr"aventionsstrategien?}
\begin{itemize}
        \item \textbf{Allgemeine Pr"avention}
                \begin{itemize}
                        \item \textbf{Vorteile}
                                \begin{itemize}
                                        \item geringe Stigmatisierung bei Risikopopulationen
                                        \item fr"uher Ansatz
                                        \item breite "Offentlichkeit
                                \end{itemize}
                        \item \textbf{Nachteile}
                                \begin{itemize}
                                        \item geringe Effektst"arken
                                        \item hohe Kosten
                                        \item hohe Stigmatisierung bei Nicht-Risikopopulationen
                                        \item hohe Anzahl an F"allen ohne Probleme (geringe Motivation)
                                \end{itemize}
                \end{itemize}
        \item \textbf{Gezielte Pr"avention}
                \begin{itemize}
                        \item \textbf{Vorteile}
                                \begin{itemize}
                                        \item Zuschnitt auf Zielgruppe
                                        \item spezifischere Hilfe m"oglich
                                        \item geringe Kosten
                                \end{itemize}
                        \item \textbf{Nachteile}
                                \begin{itemize}
                                        \item Stigmatisierung
                                        \item hohe Interventionsintensit"at notwendig
                                        \item Implementation (Dropouts, selektive ``Stichproben'')
                                \end{itemize}
                \end{itemize}
\end{itemize}

\subsubsection{Was sind Einschr"ankungen von individuellen Pr"aventionsstrategien?}
Es gibt viele Ph"anomene, die wesentlich von extraindividuellen Einfl"ussen abh"angen. Es konnte z.B. gezeigt werden, dass \emph{Ungleichheit der Einkommen} in einem starken Zusammenhang mit \emph{sozialen und gesundheitsbezogenen Problemen} steht. Da sto"sen individuelle Ans"atze an eine Grenze.
