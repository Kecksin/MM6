\subsection{Fragen und Antworten zum Text von \textcite{stephan_improving_2012}}
\subsubsection{Was ist realistische Bedrohung und was symbolische Bedrohung? Nennen Sie Beispiele?}
\emph{Realistische Bedrohung} zielt auf tats"achliche Bedrohung z.B. in Form von Wettbewerb um Jobs, w"ahrend \emph{symbolische Bedrohung} auf Werte, "Uberzeugungen, etc. abzielt. Damit kann z.B. Angst verbunden sein, dass die eigene Lebensweise bedroht ist.

\subsubsection{Welche M"oglichkeiten zur Verbesserung von Intergruppenprozessen auf Gesellschaftsebene werden gegeben? Nennen Sie Institutionen, die involviert sind!}
Das muss in den Communities selber passieren. Wichtig daf"ur sind Community Leaders. Involvierte Institutionen sind die Regierung, das Bildungsystem oder die Medien.

\subsubsection{Welche M"oglichkeiten zur Verbesserung der Intergruppenbeziehunf auf individueller Ebene werden gegeben? Erl"autern Sie kurz die drei Typen von Programmen, die in dem Artikel behandelt werden!}
Auf individueller Ebene m"ussen Vorurteile und Stereotype abgebaut werden. Die drei Typen von Programmen sind:
\begin{itemize}
        \item \textbf{Erleuchtung.} Hier geht es um Wissensvermittlung.
        \item \textbf{Kontakt.} Hier geht es um Intergruppenkontakt.
        \item \textbf{F"ahigkeiten.} Z.B. \emph{Peer Mediation Training}. Es sollen Kommunikationsf"ahigkeiten vermittelt werden.
\end{itemize}

\subsubsection{Welche vier Aspekte sollen laut Stephan bei der Implementierung beachtet werden? Welche Implikationen ergeben sich jeweils?}
\begin{itemize}
        \item Einstellungen zu "andern (vor allem von Rechtskonservativen) ist schwer
        \item Bedarfe von Communities m"ussen immer erst ermittelt werden
        \item Immigranten sollten an der Planung beteiligt sein
        \item Evaluation!
\end{itemize}



