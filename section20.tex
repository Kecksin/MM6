\subsection{Fragen und Antworten zum Text von \textcite{chen_why_2004}}

\subsubsection{Wie verl"auft der Zusammenhang zwischen sozio"okonomischem Status und Gesundheit? }
Positiv: steigender SES -> steigende Gesundheit: \\
Gradient: jede h"ohere Stufe des SES geht mit einem inkrementellen Vorteil f"ur die Gesundheit einher.

\subsubsection{Welche Erkl"arungen f"ur den Zusammenhang zwischen sozio"okonomischem Status und Gesundheit wurden vorgeschlagen? }
Forscher haben viele Erkl"arungen hierzu vorgeschlagen:
\begin{itemize}
        \item genetische Einfl"usse
        \item Umwelteinfl"usse
        \item Giftstoffe
        \item Qualit"at der medizinischen Versorgnung
        \item psychologische und verhaltensbasierte Faktoren
\end{itemize}

\subsubsection{In welche vier Kategorien lassen sich die psychologischen und verhaltensbasierten Erkl"arungsans"atze gruppieren? Nennen Sie Forschungsbefunde f"ur jede der Kategorien. }
\begin{enumerate}
        \item Stress:
                \begin{itemize}
                        \item niedrigerer SES >> mehr negative Ereignisse werden erlebt (Stressoren) >> gr"o"sere negative Auswirkungen wahrgenommen (stress appraisal)
                        \item Stress ist ein plausibler Mediator in der Beziehung zwischen SES und Gesundheit
                        \item Vermutung: wenn der SES einer Person sinkt, nimmt die Menge an erlebtem Stress zu >> wirkt sich wiederum als Tribut auf den K"orper aus
                \end{itemize}
        \item Psychische Belastung:
                \begin{itemize}
                        \item niedrigerer SES >> anf"alliger f"ur das Erleben negativer emotionaler Zust"ande >> biologische Konsequenzen >> ggf. schlechtere Gesundheit
                \end{itemize}
        \item Pers"onlichkeitsfaktoren
                \begin{itemize}
                        \item Personen mit niedrigerem SES besitzen Pers"onlichkeitseigenschaften, die sch"adlich f"ur die Gesundheit sind
                        \item Bspw.: Misstrauen, zynische Haltung, Feindseligkeit, weniger Optimismus >> erh"ohtes Risiko f"ur Krankheiten
                \end{itemize}
        \item Gesundheitsverhalten 
                \begin{itemize}
                        \item niedriger SES >> weniger wahrscheinlich, dass gesunde Verhaltensweisen wie Sport, gesunde Ern"ahrung und Nichtrauchen ausge"ubt werden
                        \item kann an der Verf"ugbarkeit von Ressourcen liegen
                        \item Personen mit eingeschr"anktem Zugang zu gesunden Produkten in ihrer Nachbarschaft werden Schwierigkeiten bei einer gesunden Ern"ahrung haben 
                \end{itemize}
\end{enumerate}

ACHTUNG: Die meisten obigen Faktoren fokussieren auf das Individuum. Beim Versuch die Gesundheit von Kindern zu verstehen, ist es besonders wichtig die Rolle der Familie und der weiteren Umgebung zu betrachten

\subsubsection{Welche drei Modelle versuchen die Entwicklung des Zusammenhangs zwischen sozio"okonomischem Status und Gesundheit in Kindheit und Jugendalter zu beschreiben und wie erkl"aren sie den vorgeschlagenen Verlauf?}
\begin{enumerate}
        \item \textbf{Childhood-limited Model} Beziehung zwischen SES und Gesundheit ist am st"arksten in der fr"uhen Kindheit und wird mit dem Alter schw"acher. Wichtig: Qualit"at der Kinderbetreuung, Bindung an die Eltern, Wohnverh"altnisse
\item \textbf{Adolescent-emergent Model} Beziehungen zwischen SES und Gesundheit sind in der Kindheit schwach und werden mit dem Alter st"arker. Wichtige Faktoren in der Adoleszenz: Peer-Einfluss, bestimmte Pers"onlichkeitsmerkmale
\item \textbf{Persistence Model} Beziehung zwischen SES und Gesundheit ist in Kindheit und Jugendalter "ahnlich 
\end{enumerate}

\subsubsection{Neben der individuellen Ebene k"onnen auch andere Ebenen zur Erkl"arung des Zusammenhangs zwischen sozio"okonomischem Status und Gesundheit herangezogen werden. Welche sind dies und wie k"onnten sie von Bedeutung sein?}
\begin{itemize}
        \item \textbf{Sozialpolitik.} Unterschiedliche Gesellschaften k"onnen verschiedene Ebenen des Vertrauens und des Zusammenhalts zwischen den Communitymitgliedern haben und unterschiedliche Investitionen in die Community f"ordern (Social Capital). Forschung zeigt: Social Capital mediiert den Zusammenhang zwischen SES und Gesundheit.
        \item \textbf{Nachbarschaftsebene.} Hier spielen mehrere Faktoren eine Rolle:
                \begin{itemize}
                        \item gef"ahrliche Nachbarschaften: Barrieren um gesundheitsrelevantes Verhalten auszu"uben
                        \item toxische Umgebungen 
                        \item Grad an Segregation: Segregierte Nachbarschaften investieren tendenziell weniger in "offentliche Dienstleistungen als integrierte Nachbarschaften
                \end{itemize}

        \item \textbf{Familie.} Qualit"ot der Beziehungen innerhalb der Familie. Familien mit vielen Konflikten und kalten, nicht-unterst"utzenden Beziehungen haben eher Kinder, die lebenslang unter gesundheitlichen Problemen leiden und fehlregulierte biologische Systeme aufweisen. 
\end{itemize}

\subsubsection{Welche zwei Bereiche sollten in zuk"unftiger Forschung genauer untersucht werden?}
\begin{enumerate}
        \item st"arker integriertes Verst"andnis des Mechanismus, der hinter der Beziehung von SES und Gesundheit steht durch Ber"ucksichtigung von \emph{gesellschaftlichen}, \emph{Nachbarschafts-}, \emph{Familien-} und \emph{individuellen Variablen}.
        \item dynamische Effekte des SES auf die k"orperliche Gesundheit: welche Art von Auswirkungen auf die Gesundheit sind von welchen Eigenschaften des SES abh"angig?
\end{enumerate}







	
