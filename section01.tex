\section{Zentrale Theorien in der Community Psychologie}

\subsection{Fragen und Antworten zum Text von \textcite{prilleltensky_value-based_2001}}
\subsubsection{Wie begr"undet Prilleltensky, dass Fragen in Bezug auf Werte im Vergleich zu methodischen Fragen weniger spezifisch und weniger ausgereigt formuliert sind?}
Wissenschaftliche und akademische Traditionen verlangen die Konzentration auf wissenschaftliche, nicht auf moralische Aspekte.

\subsubsection{Was sind zentrale Merkmale der Community Psychologie und wie unterscheiden sie sich von der traditionellen angewandten Psychologie?}
Die Community Psychologie wurde in den $1960$er Jahren ins Leben gerufen als Reaktion auf Probleme und Vers"aumnisse vor allem der klinischen Psychologie. F"ur eine Gegen"uberstellung der Ansatzpunkte siehe Fig. \ref{fig:prilleltensky1}. Besonders hervorzuheben sind zwei Punkte:

\begin{itemize}
        \item \textbf{Normen.} Community Psychologie schlie"st ausdr"ucklich die Frage nach Normen mit ein, die von der traditionellen Psychologie (und Wissenschaft im Allgemeinen) ausgeschlossen wird.
        \item \textbf{Soziale Ver"anderung.} Aus der Besch"aftifgung mit Normen entsteht die Forderung nach sozialer Ver"anderung. Daher auch die Notwendigkeit einer "okologischen Perspektive. Soziale Vera'nderung ist besonders schwer zu erreichen, weil sie den Status Quo angreift.
\end{itemize}

\begin{figure}[tbh]
        \begin{center}
                \begin{tikzpicture}[
                        comm/.style={shape=rectangle,draw,rounded corners,fill=green!20,text width=3.5cm,text centered,node distance=0.7cm},
                        klin/.style={shape=rectangle,draw,rounded corners,fill=red!20,text width=3.5cm,text centered,node distance=0.7cm}
                        ]
                        \node[comm] at (-4,+0) (prev) {Pr"avention};
                        \node[comm] [below =of prev] (eco) {Kontext};
                        \node[comm] [below =of eco] (self) {Selbsthilfe};
                        \node[comm] [below =of self] (coll) {Kollaboration};
                        \node[comm] [below =of coll] (prom) {F"orderung von Wohlbefinden};
                        \node[klin]at (+4,+0) (kur) {Kuration};
                        \node[klin][below =of kur] (ind) {Indiviuum};
                        \node[klin][below =of ind] (pro) {Professionelle Hilfe};
                        \node[klin][below =of pro] (exp) {Expertentum};
                        \node[klin][below =of exp] (diag) {Diagnose von Krankheiten};
                \end{tikzpicture}
                \caption{Unterschiede zwischen Community Psychologie und klinischer, bzw. traditioneller angewandter Psychologie.}
                \label{fig:prilleltensky1}
        \end{center}
\end{figure}

\subsubsection{Welches Ziel wurde bisher nicht (ausreichend) erreicht? Welche Barrieren haben es verhindert?}
Nicht ausreichend erreicht worden ist bisher \emph{Soziale Gerechtigkeit}. Das erfordert n"amlich soziale Ver"anderung, und daran sind diejenigen, die an der Macht (oder allgemeiner: gut gestellt) sind, nicht interessiert. 

\subsubsection{Wie lautet die Definition von Wellness?}
Wellness ist nicht die Abwesenheit von Krankheit, sondern eine positive Ausgangslage, die sich im Vorhandensein von psychologischen und materiellen Ressourcen zeigt.

\subsubsection{Welchen vier Kriterien sollten Werte entsprechen?}
\begin{itemize}
        \item \textbf{Richtlinien f"ur den Weg zum Idealzustand} Kongruenz von Weg und Ziel.
        \item \textbf{Kein Dogma oder Relativismus} Man sollte sich also weder auf etwas versteifen, noch alles relativieren.
        \item \textbf{Komplementarit"at} Die Werte, auf die man sich beruft, sollten zueinander passen.
        \item \textbf{Werte sollen personliches, kollektives und relationales Wohlbefinden f"ordern}
\end{itemize}

\subsubsection{Nennen sie Beispiele f"ur Werte, die einander erg"anzen!}
Diversit"at und Selbstbestimmtheit, Kollaboration und demokratische Partizipation, Sorge und Gerechtigkeit.

\subsubsection{Wie kann die Beziehung zwischen pers"onlichem und kollektivem Wohlergehen beschrieben werden? Nennen sie Beispiele f"ur pers"onliche und kollektive Ziele, die im Widerspruch zueinander stehen!}
Pers"onliches Wachstum und Soziale Gerechtigkeit k"onnen im Widerspruch zueinander stehen, genauso wie Selbstbestimmung und Kollaboration. Au"serdem kann Selbstbestiimung (Rauchen) in Konflikt mit Gesundheit stehen.

\subsubsection{Was ergibt sich als Konsequenz aus den m"oglichen Widerspr"uchen?}
Es muss ein \emph{Relationales Wohlbefinden} geben, dass zwischen pers"onlichem und kollektivem Wohlbefinden vermittelt.

\subsubsection{Was muss beim Wert \emph{support for enabling community structures} beachtet werden?}
Dass nicht individuelle Rechte beschnitten und Konformit"at und Uniformit"at gef"ordert werden.

\subsubsection{Was muss beim Wert \emph{respect for diversity} beachtet werden?}
Dass Solidarit"at nicht zu kurz kommt und dass es nicht zu sozialer Fragmentierung kommt.

\subsubsection{Was ist neben der F"orderung von Wohlergehen eine weitere Funktion von Werten?}
Pr"avention von Problemen. Fahrverbot mit Alkohol verhindert Unf"alle. Rauchverbot an "offentlichen Pl"atzen verhindert Passivrauchen.

\subsubsection{Auf welcher Handlungsebene liegt der Fokus der meisten Pr"aventionsprogramme? Welcher Wert kommt dadurch zum Ausdruck?}
Auf der Individualebene. Dadurch werden vor allem \emph{Selbstbestimmtheit} und \emph{Pers"onliches Wachstum} betont.

\subsubsection{Welche Rolle spielt kollektives Wohlergehen in der Agenda der Community Psychologie?}
Obwohl con der CP gefordert, steht kollektives Wohlergehen in der Praxis im Hintergrund. Es wird zwar in Bezug auf bestimmte abgegrenzte Themen daran gearbeitet (Drogenmi"sbrauch, Depression), was aber soziale Gerechtigkeit angeht, passiert nicht viel. 

\subsubsection{Inwiefern m"ussen Werte kontextspezifisch angepasst werden?}
Nicht alle Werte sind in allen Situationen g;eich wichtig. Au"serdem k"onnen Werte f"ur unterschiedliche Gruppen unterschiedliche Bedeutung haben. F"ur Behinderte ist der Wert der \emph{Unabh"angigkeit} beispielsweise nicht unbedingt im Fokus.

\subsubsection{Was sind die Phasen des Praxis-Kreislaufs? Welche zentralen Kriterien sind relevant?}
Die Phasen sind \emph{Reflektion}, \emph{Forschung} und \emph{Handeln} (Siehe Fig. \ref{fig:prilleltensky2}). Die 4 Kriterien sind:
\begin{itemize}
        \item \textbf{Philosophie vs. Realismus} Induktion und Deduktion. Nur abstrakte philosophische "Uberlegungen bringen nichts.
        \item \textbf{Verstehen vs. Handeln} Damit es nicht nur intellektueller Diskurs bleibt.
        \item \textbf{Prozesse vs. Ourcomes} Damit es nicht zum Selbstzweck wird.
        \item \textbf{Alle Stimmen sollen geh"ort werden} Damit nicht nur die Politiker zu Wort kommen.
\end{itemize}
\begin{figure}[hb!]
        \begin{center}
                \begin{tikzpicture}[mindmap, every node/.style={concept,execute at begin node=\hskip0pt},grow cyclic,font=\large\scshape,concept color=black!20,
                level1/.append style={level distance=4cm,angle=120}]
                        \node {Praxis Kreislauf}
                        child{node{Reflektion}}
                        child{node{Forschung}}
                        child{node{Handeln}};
                \end{tikzpicture}
        \end{center}
        \caption{Der Praxis Kreislauf von \textcite{prilleltensky_value-based_2001}.}
        \label{fig:prilleltensky2}
\end{figure}

\subsubsection{Warum ist eine Kontextanalyse wichtig?}
Um die subjektive Erfahrungswelt der Mitglieder der Community kennenzulernen. Es gibt z.B. Unterschiede zwischen individualistische und kollektivistischen Kulturen hinsichtlich Normen, Trends, etc. Diese m"ussen identifiziert und ber"ucksichtigt werden.


