\subsection{Fragen und Antworten zum Text von \textcite{dixon_beyond_2012}}
\subsubsection{Skizzieren sie die beiden Wegen zu sozialer Ver"anderung in Gesellschaften, in denen Ungleichheit herrscht. Stellen sie die beiden Modelle einander anhand der Taxonomie von \textcite{dixon_beyond_2012} gegen"uber. Worin liegen die Unterschiede hinsichtlich der haupts"achlich Agierenden? Welche Konsequenzen ergeben sich f"ur die Intervention? Welche psychologischen Prozesse werden jeweils posutliert? Welche Konzequenzen ergeben sich auf Verhaltensebene?}

Siehe dazu fig. \ref{fig:dixon1}.

\begin{figure}[hb!]
        \begin{center}
                \begin{tikzpicture}[scale=0.8,
                        mod1/.style={shape=rectangle,draw,fill=red!20,text width=3cm,text centered,rounded corners,minimum size=1.5cm},
                        mod2/.style={shape=rectangle,draw,fill=green!20,text width=3cm,text centered,rounded corners,minimum size=1.5cm}
                        ]
                        %Prevention Reduction Model
                        \node [font=\large\scshape] at (0,+4) (mod1) {Prejudice Reduction Model};
                        \node [mod1] at (-6,+2) (memb1) {Mitglieder der bevorteilten Gruppe};
                        \node [mod1,right=of memb1] (int1) {Intergruppenkontakt};
                        \node [mod1,right=of int1] (proc1) {Reduzierung von Stereotypen};
                        \node [mod1,right=of proc1] (out1) {Reduktion von Intergruppenkonflikt};
                        %Collective Action Model
                        \node [font=\large\scshape] at (-0,-0) (mod2) {Collective Action Model};
                        \node [mod2] at (-6,-2) (memb2) {Mitglieder der benachteiligten Gruppe};
                        \node [mod2,right=of memb2] (int2) {Empowerment};
                        \node [mod2,right=of int2] (proc2) {Ungerechtig\-keits\-ge\-f"uhl};
                        \node [mod2,right=of proc2] (out2) {Kollektive Aktionen zur "Anderung des Status Quo};
                \end{tikzpicture}
        \end{center}
        \caption{Die zwei Wege zu sozialer Ver"anderung nach \textcite{dixon_beyond_2012} gegen"ubergestellt. Das eher traditionelle Prevention Redction Model in rot, das Collective Action Model in gr"un. In den K"astchen stehen Agierende, Interventionsarten, psychologische Prozesse und Konsequenzen auf Verhaltensebene. }
        \label{fig:dixon1}
\end{figure}

\subsubsection{Was ist mit den paradoxen Effekten von Intergruppenkontakt gemeint?}
Intergruppenkontakt hat zwar einen positiven Effekt auf Einstellungen der bevorteiligten Gruppe, aber auch einen negativen Effekt auf wahrgenommene soziale Ungerechtigkeit bei der benachteiligten Gruppe. Damit einher geht auch verringerte Bereitschaft zu kollektivem Handeln.

\subsubsection{Welche Schlussfolgerungen werden in Bezug auf die Vereinbarkeit der beiden Modelle gezogen?}
Beide Modelle haben unterschiedliche Effekte. Die Effekte des einen untergraben dabei den Sinn des anderen Modells.

\subsubsection{W"ahlen sie drei der Kommentare aus und fassen sie jeweils die Hauptaussagen zusammen!}
\paragraph{Kommentar von Alicke: You say you want a revolution?}
Zwei Kritikpunkte:\\
\begin{itemize}
        \item \textbf{Einstellungen und Diskriminierendes Verhalten werden gleichgesetzt.} Die Autoren behaupten, dass Einstellungen quasi automatisch zu einem bestimmten Verhalten (wie Diskriminierung) f"uhren. Das muss aber nicht unbedingt so sein und wird auch durch zahlreiche sozialpsychologische Theorien zum Zusammenhang von Einstellungen und Verhalten konterkariert.
        \item \textbf{Keine Evidenz, dass Reduktion von Vorurteilen nicht zu weniger Diskriminierung f"uhrt.} Wenn es stimmt, was die Autoren sagen, dann m"usste es so sein, dass es bei einer Reduktion von Vorurteilen \emph{nicht} zu einer Reduktion von Diskriminierung kommt. Diesen Beweis bleiben die Autoren aber schuldig.
\end{itemize}

\paragraph{Kommentar von Bilewicz: Traditional prejudice remains outside of the WEIRD world}
Subtile Formen von Vorurteilen treten vor allem in WEIRD (Western, Educated, Industrialized, Rich, Democratic) Nationen auf, wo es starke Normen f"ur politische Korrektheit und damit gegen offene Vorurteile gibt. Dementsprechend ist der Zusammenhang von Vorurteilen und Diskriminierung auch nicht besonders stark. In ``Nicht-WEIRD'' Staaten dagegen gibt es immer noch offene Vorurteile, und diese sind auch ein guter Pr"adiktor f"ur tats"achliche Diskriminierung. Vorurteilsreduktion w"are in diesen Staaten ein durchaus wirksamer Weg.

\paragraph{Kommentar von Charles et al.: Insights from studying prejudice in the context of American atheists}
Die Autoren unterst"utzen die Erforschung von neuen Formen von Vorurteilen. Forschung mit Atheisten zeigt dabei, dass die von Dixon et al. vorgebrachten Formen nicht in jedem Kontext passen, wie z.B. im Zusammenhang mit Atheisten.
\begin{itemize}
        \item \textbf{Non-Religion ist anders.}
                \begin{itemize}
                        \item Diskriminierung erfordert Coming-out
                        \item Non-Religi"osit"at ist nicht salient
                        \item Vorurteile gegen Nicht-Relig"ose sind ausschlie"slich negativ und nicht ambivalent
                        \item es gibt keine gemeinsame Identit"at von Religi"osen und Nicht-Religi"osen 
                \end{itemize}
        \item \textbf{Probleme f"ur die Vorurteilsreduktion.} Weil sie oft unerkannt bleiben, kann man auch schlecht Ma"snahmen zum Intergruppenkontakt einleiten.
        \item \textbf{Das Ph"anomen offenbart interessante Fragestellungen.}
                \begin{itemize}
                        \item Was sind Rechtfertigungen f"ur das Verstecken oder "Offentlichmachen?
                        \item Welche Ma"snahmen f"ur die Bevorurteilten gibt es?
                \end{itemize}
        \item \textbf{Verstecken ohne sich zu verstecken.} Nicht-Religi"ose, die sich nicht geoutet haben, sind oft nicht der Meinung, dass sie sich verstecken w"urden.
        \item \textbf{Wer hat hier die Vorurteile?} H"aufig outen sich Nicht-religi"ose nicht aus ``R"ucksicht'' z.B. vor den Gro"seltern. Hier findet sich eine verkehrte Art des \emph{Paternalismus} von Dixon et al. 
\end{itemize}
