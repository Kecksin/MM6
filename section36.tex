\subsection{Fragen und Antworten zum Text von \textcite{morton_applying_2012}}
\subsubsection{Was versteht man unter \emph{geographic information systems}?}
Eine Kombination von \emph{tools} zur Analyse von r"aumlichen Informationen. Diese werden dabei in Verbindung gebracht mit anderen (z.B. soziodemografischen) Informationen.

\subsubsection{Warum kann es sinnvoll sein, GIS in der Community Forschung einzusetzen?}
Eine Pr"amisse der Community Psychologie ist es, dass Menschen nicht ohne Ber"ucksichtigung der Kontexte verstanden werden k"onnen, in denen sie leben. Damit sind auch geografische Kontexte gemeint.

\subsubsection{Was sollte beim Einsatz von GIS beachtet werden?}
Es gibt das Problem der r"aumlicher Autokorrelation. das fr"uher als das ``erste Gesetz der Geografie'' bekannt geworden ist: Alles steht in Verbindung mit allem, aber Nahes ist st"arker verbunden als Entferntes.

\subsubsection{Welche Arten von Daten k"onnen in GIS einflie"sen?}
Prim"are (selbsterhoben) und sekund"are (z.B. Census) Daten.

\subsubsection{Erkl"aren sie die unterschiedlichen Darstellungsweisen, die mit GIS m"oglich sind!}
\begin{itemize}
        \item \textbf{Vector Model}
                \begin{itemize}
                        \item 2-dimensionale Informationen 
                        \item Punkte auf der Karte
                        \item Verknuepfung zu Linien oder Vielecken
                        \item diskrete Daten
                \end{itemize}
        \item \textbf{Raster Model}
                \begin{itemize}
                        \item Raster "uber einer Fl"ache 
                        \item innerhalb einer Zelle kann etwas dargestellt werden
                        \item nicht so pr"azise wie Vector Models
                        \item kontinuierliche Daten
                \end{itemize}
\end{itemize}

\subsubsection{Welche Aspekte sollten bei der Datenauswertung beachtet werden?}
Grunds"atzlich werden die Ergebnisse als \emph{thematische Karten} visualisiert. Man unterscheidet 3 Formen:
\begin{itemize}
        \item \textbf{Chloropleth (Graduated Colour Map).} Unterschiedliche Werte werden durch Farben repr"asentiert
        \item \textbf{Graduated Symbol Map.} Unterschiedliche Werte werden durch unterschiedliche Symbole repr"asentiert
        \item \textbf{Dot Density Map.} Position eines Merkmals innerhal eines Gebiets 
\end{itemize}

\subsubsection{Beschreiben sie die dargestellte Studie, die im Rahmen des C.O.P.E.-Projekts durchgef"uhrt wurde. Warum war es sinnvoll, mit GIS zu arbeiten?}
In einer Gemeinde wurden alle \emph{Pharmacies} auf einer Karte dargestellt. Manche verkaufen auch sch"adliche Substanzen (Alkohol), andere nicht. Es wurde untersucht, ob es in Gebieten mit hohem Migrantenanteil mehr ``b"ose'' Pharmacies gibt. 

\subsubsection{Was sind Vor- und Nachteile der Verwendung von GIS in der Community Forschung?}
Der gro"se Vorteil ist sicher, dass die selbstgesteckten Anforderungen an Kontextbezogenheit damit erf"ullt werden. Das erm"oglicht auch ``passgenaue'' Interventionen. Als Nachteile werden genannt:
\begin{itemize}
        \item Kosten f"ur Software und Training (?!?)
        \item Datenqualit"at (der meist ben"otigten sekund"aren Daten)
\end{itemize}
