\subsection{Fragen und Antworten zum Text von \textcite{sue_racial_2007}}
\subsubsection{Wie ist der Begriff der \emph{Mikroaggressionen} definiert?}
Mikroaggressionen sind kurze verbale, nonverbale oder behaviourale Dem"utigungen. Sie k"onnen intentional oder nicht intentional sein, ebenso wie bewusst oder unbewusst.

\subsubsection{Beschreiben sie die drei Formen von Mikroaggressionen!}
\begin{description}
        \item[Microassault] Ein \emph{Mikroangriff} ist ein expliziter Angriff mit der Intention, die Zielperson zu verletzen. Dementsprechend handelt es sich um bewusste Handlungen, z.B. wenn jemand die Stra"senseite wechselt, weil ihm ein Migrant entgegenkommt.
        \item[Microinsult] Eine \emph{Mikrobeleidigung} zeigt sich in fehlender Sensibilit"at und Herablassung gegen"uber Rasse oder ethnischer Identit"at einer Person. Mikrobeleidigungen k"onnen subtil und f"ur den Beleidgenden unbewusst sein. F"ur den Beleidigten sind sie in aller Regel durchaus bewusst, z.B. wenn ein Deutscher mit Migrationshintergrund wiederholt danach gefragt wird, woher er kommt.
        \item[Microinvalidation] Wenn Gef"uhle, Gedanken oder erlebte Realit"at von Migranten heruntergespielt, negiert oder ganz ignoriert werden. Z.B. ``Alle Menschen sind gleich.''
\end{description}

\subsubsection{Welche vier Dilemmata bestehen im Hinblick auf Mikroaggressionen?}
\begin{itemize}
        \item Unterschiede in der Wahrnehmung von ethnischen Realit"aten
        \item Unsichtbarkeit von nicht intendierten Mikroaggressionen
        \item Wahrgenommener minimaler Schaden
        \item Teufelskreis bei der Reaktion auf Mikroaggressionen
\end{itemize}

\subsubsection{Dilemma 1: Welche Unterschiede bestehen zwischen \emph{people of colour and Whites}?}
\emph{Whites} sind der Meinung, dass es Rassismus effektiv nicht mehr oder kaum noch gibt, und dass Migranten genauso wie sie oder evtl. sogar besser behandelt werden. Die erlebte Realit"at von \emph{people of colour} sieht dagegen ganz anders aus.

\subsubsection{Dilemma 2: Welche Rolle spielt \emph{Colour Blindness}?}
\emph{Farbenblindheit} ist eine schwere Form von Mikroinvalidierung. \emph{Whites} verhalten sich faktisch diskriminierend, beteuern aber, dass ihr Handeln nichts mit der Hautfarbe zu tun habe. Tats"achlich ist es m"oglich, dass ihnen ihr diskriminierendes Verhalten nicht bewusst ist.

\subsubsection{Dilemma 3: Wie werden die Auswirkungen von Mikroaggressionen "ublicherweise eingesch"atzt?}
Sie werden als nicht schwerwiegend eingesch"atzt. Oft ist sich der Diskriminierende ja nicht einmal bewusst, dass er diskriminiert hat. Tats"achlich spricht aber vieles daf"ur, dass Mikroaggressionen schlimmeren Schaden anrichten als traditionelle, offene Aggressionen. Und zwar weil f"ur das Opfer die Frage bleibt, ob es sich um eine Beleidigung gehandelt hat, ob er reagieren soll und wenn ja, wie.

\subsubsection{Dilemma 4: Welche Rolle spielen bisherige Erfahrungen in der Beurteilung von Handlungen als Mikroaggressionen?}
Wie die Frage suggeriert eine Gro"se. Bemerkungen, die als Mikroaggressionen interpretiert werden k"onnen, m"ussen nicht unbedingt solche sein. Es kommt auf den Kontext an. F"ur einen Migranten, der immer wieder Mikroaggressionen ausgesetzt ist, wird ein Verhalten aber eher als rassistisch empfunden, wenn er solches oder "ahnliches Verhalten in anderen Situationen auch schon erlebt hat. Das steht im Gegensatz zum Empfinden des Aggressors, der nur sein Verhalten in der fraglichen Situation kennt, sich also des h"aufigen Vorkommens solcher Situationen nicht bewusst ist.

\subsubsection{Inwiefern spielen Mikroaggressionen eine Rolle im Kontext von Beratung und Therapie?}
F"ur die Beziehung zwischen Klient und Therapeut ist gegenseitiges Vertrauen essentiell. Durch Mikroaggressionen wird dieses Vertrauen gest"ort und kann die Beratung oder Therapie torpedieren.

\subsubsection{Welcher Aspekt von Mikroaggressionen ist die gr"o"ste Herausforderung f"ur Angeh"orige der Majorit"at im Gesundheitswesen?}
Das Unsichtbare sichtbar zu machen. Dies kann nur vollf"uhrt werden, wenn Menschen bereit sind offen und ehrlich den Rassendialog zu f"uhren. Wenn Therapeuten Schwierigkeiten haben das Thema anzusprechen, verschlie�en sie Klienten den Weg dazu, Bereiche der Verzerrung, Diskriminierung und Vorurteile zu erforschen.

