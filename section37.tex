\subsection{Fragen und Antworten zum Text von \textcite{trickett_how_2011}}
\subsubsection{Wie wurde ``Community Intervention'' definiert und welche Studien wurden mit einbezogen?}
Folgende Kriterien wurden angelegt:
\begin{itemize}
        \item Community-placed
        \item Community-based
        \item pr"aventiv
        \item unterst"utzend
        \item Verhaltens"anderung intendiert
\end{itemize}

Alle Studien, die von Forschern initiiert wurden, oder bei denen die Forscher angefragt wurden sind ber"ucksichtigt worden.

\subsubsection{Wie lauten die Forschungsfragen f"ur die Untersuchung? Wie wurden die Studien ausgew"ahlt und wie die Kodierungen erstellt?}
Die 2 Forschungsfragen lauteten:
\begin{itemize}
        \item Wie beschreiben die Autoren ihre Intervention?
        \item Was kann man daraus in Bezug auf die Konzeptualisierung von Interventions schlie"sen?
\end{itemize}
Es wurden alle Studien miteinbezogen, die gesunheitsf"orderliche Interventionsma"snahmen in Beuzg auf \emph{Prozess} oder \emph{Outcome} beinhalten und den oben beschriebenen Kriterien gen"ugen.

Die Kategorien wurden auf Basis eines \emph{Grounded Theory}- Ansatzes erstellt, Kategorien also entstammen den Texten. Zus"atzlich wurden Kategorien aus \emph{bereits exisitierender Forschung} zu Community Interventionen ber"ucksichtigt. Au"serdem wurden \emph{Interviews} mit Autoren von $10$ nominierten Forschungsartikeln gef"uhrt.

\subsubsection{Welche f"ung Kategorien wurden erstellt und was wurde darin kodiert? Was waren die Ergebnisse?}

\begin{itemize}
        \item Fokus und Erfolg
                \begin{itemize}
                        \item Einsatzgebiet (Lifestyle, Substanzmissbrauch,\ldots)
                        \item Erfolg der Intervention (Ja, Nein, Partiell)
                \end{itemize}
        \item Kontext und Stichprobe
                \begin{itemize}
                        \item Kontext (beschrieben: Ja/Nein)
                        \item Vpn (beschrieben: Ja/Nein)
                        \item Diversity
                \end{itemize}
        \item Design Merkmale
                \begin{itemize}
                        \item Random Assignment
                        \item Interventionskomponenten (erw"ahnt)
                        \item Interventionstreue (wurde erfasst)
                        \item Intentsit"atsrechtfertigung (Ja, Nein)
                        \item Nachhaltigkeit (erw"ahnt)
                \end{itemize}
        \item Theorie und Ziele
                \begin{itemize}
                        \item Theorie (explizit)
                        \item Analyseebene (Individuell)
                        \item Interventionsziele (Individuell/Extraindividuell)
                \end{itemize}
        \item Community und Interventionsprozess
                \begin{itemize}
                        \item Umweltbezogene Erfassung (Kennenlernen der Community)
                        \item Begr"undung der Intervention (wissenschaftlich, wissenschaftlich + Community, Community)
                        \item Community Kollaboration (Involvierung, in Implementation, in Nachbesserung, 
                        \item Feedback
                        \item Abschlussaktivit"aten
                \end{itemize}
\end{itemize}

\subsubsection{Wie unterschieden sich Studien, die f"ur die individuelle Ebene konzipiert wurden von Interventionen, die auch oder nur auf "ubergeordnete Ebenen abzielen?}
Interventionen, die auf Individuen abzielen, sehen Nachhaltigkeit vor allem in Bezug auf F"ahigkeiten oder Infrastruktur, nicht in Bezug auf das Programm. In diesem Studien gab es auch insgesamt weniger Partizipation. Au"serdem wurden in den Individualstudien noch weniger als sowieso schon \emph{closure}-Aktivit"aten durchgef"uhrt, also Gedanken zu anhaltenden Effekten auch nach Beendigung der Studie.

\subsubsection{Was fanden und diskutieren \textcite{trickett_how_2011} bez"uglich Einbeziehen der Community und Umgang mit Diversity?}
Die Mehrheit der Studien involviert die Mitglieder der Community, aber meistens nur, um den Forschern zu helfen. Die Mehrheit der Studien wurde au"serdem mit diversen Stichproben durchgef"uhrt. Diese Diversit"at wurde allerdings nicht ber"ucksichtig. Nicht bei der \emph{theoretischen Planung der Intervention}, nicht bei \emph{"Uberlegungen zur Stichproben} und auch nicht bei der \emph{Datenanalyse}. Implizit liegt also die Annahme zu Grunde, dass die Interventionen universell wirksam waren, also f"ur alle Gruppen in gleicher Weise. 

\subsubsection{Was diskutieren die Autorinnen in Bezug auf Nachhaltigkeit?}
Nur $30 \%$ der Studien erw"ahnen Nachhaltigkeit. Nur $\frac{1}{3}$ davon berichten "uber (fehlende) Evidenz von Nachhaltigkeit. 

\subsubsection{Weswegen sollten Interventionstheorien komplexer und kontextsensitiver werden? Was sollte in ihnen enthalten sein?}
Weil Community Settings gepr"agt sind durch multiple Komponenten,  multiple Beziehungen und multiple Effekte.
