\section{Ausgew"ahlte Methoden der Community Psychologie}

\subsection{Fragen und Antworten zum Text von \textcite{jacquez_youth_2012}}
\subsubsection{Welche direkten und indirekten Vorteile hat Community-based Participatory Research (CBPR) f"ur Forschung im Bereich Gesundheit und Forschung in benachteiligten Communities?}
\begin{itemize}
        \item Fokussierung auf praktische Probleme einer Community
        \item kontextuelle Faktoren k"onnen aufgedeckt werden
        \item Empowerment von Benachteiligten
\end{itemize}

\subsubsection{Unter welchen Umst"anden - sagt Zimmerman (2005) - k"onnen auch j"ungere Jugendliche sinnvoll in Forschungsprojekte einbezogen werden?}
Die Materialien m"ussen angemessen gestaltet und dargeboten werden.

\subsubsection{Worauf haben die Autorinnen besonderen Wert gelegt in Bezug auf die einzebziehenden Studien?}
Es wurden nur Artikel aus refereed (?) Journals einbezogen, die ausdr"ucklich CBPR gemacht haben, das also auch als CBPR gekennzeichnet haben.

\subsubsection{Nur $15\%$ der gefundenen Artikel zu CBPR wurden in die Studie miteinbezogen. Warum?}
Weil nur bei $15\%$ der Studien Jugendliche als Kooperationspartner eingebunden waren.

\subsubsection{Welche f"unf Phasen eines CBPR-Forschungsansatzes wurden zugrunde gelegt, um den Grad der Involviertheit der Jugendlicher einzusch"atzen?}
\begin{enumerate}
        \item Partnerschaftsformierung und -auftrechterhaltung
        \item Communtiy Erfassung und Diagnose
        \item Definition des Problems
        \item Dokumentation und Evaluation des partnerschaftlichen Prozesses
        \item Feedback, Interpretation, Verbreitung und Anwendung der Ergebnisse
\end{enumerate}

\subsubsection{Die Autorinnen stellen fest, dass $27\%$ der gefunden Studien allenfalls \emph{community placed} und nicht einmal \emph{communty partnered} sind. Welche Konsequenzen hat es, dass in diesen Studien keine Partnerschaft mit den Mitgliedern der Community stattgefunden hat?}
\begin{itemize}
        \item eher kein Empowerment
        \item Nachhaltigkeit gef"ahrdet
        \item Externe Validit"at gef"ahrdet
\end{itemize}

\subsubsection{Auch CBPR mit Erwachsenen, die sich mit Problemlagen Jungendlicher befassen, kann effektiv sein. CBPR mit Jugendlichen hat dar"uber hinaus eigene Vorteile. Welche sind das?}
\begin{itemize}
        \item Empowerment
        \item Ergebnisse werden von (anderen) Jugendlichen eher angenommen
        \item ``Insider-Perspektive''
\end{itemize}

\subsubsection{Welche Forschungsmethoden kamen bei CBPR mit Jugednlichen zum Einsatz? Was hat sich auch f"ur j"ungere Jugendliche/Kinder bew"ahrt?}
F"ur J"ungere hat sich besonders der Moasic-Ansatz bew"ahrt, der verbale und visuelle Elemente enth"alt. Andere Methoden sind:
\begin{itemize}
        \item Fragebogenerstellung
        \item Photovoice
        \item Mosaic-Ansatz (multimethodal, verbale und visuelle Elemente)
\end{itemize}

\subsubsection{``Girls study girls'': Welches Fazit ziehen die Mitarbeiterinnen der evaluierenden Organisation im Nachgang zur Partizipativen Evaluation?}
Abgesehen von einem generell positiven Fazit hat sich gezeigt, dass die Girls nicht nur Expertinnen ihrer Belange waren, sondern auch Investigatoren in Bezug  auf Forschungsfragen. Die Frage ist nicht so sehr, was Girls eines bestimmten Alters k"onnen, sondern wie man ihre M"oglichkeiten erweitern kann. 

\subsubsection{``OPT4College'': Mit welchen Herausforderungen hatten es die Forscherinnen in ihrem iterativen Forschungsprozess zu tun?}
Die produzierten Videos wurden von den Jugendlichen geradezu zerrissen und mussten komplett neu gemacht werden.

\subsubsection{Welche Probleme identifizieren die Autorinnen, die sich bei der Untersuchung aufgrund des noch neuen und unterschiedlich verstandenen Begriffs \emph{CBPR} ergeben haben?}
Es wurden eventuell nicht alle relevanten Studien miteinbezogen. Das liegt daran, dass sich der Begriff CBPR noch nicht "uberall durchgesetzt hat. Unterschiedliche Forschungstraditionen haben zudem unterschiedliche Begriffe f"ur "ahnliche Ans"atze.


