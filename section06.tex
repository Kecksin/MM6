\subsection{Fragen und Antworten zum Text von \textcite{cattaneo_process_2010}}
\subsubsection{Durch welche zwei zentralen Merkmale l"asst sich das Empowerment-Prozessmodell von Begriffen wie \emph{mastery}, \emph{self-efficacy}, \emph{self-advocacy}, \emph{self-determination} und \emph{self-regulation} unterscheiden?}
Die zwei zentralen Merkmale sind \emph{pers"onlich bedeutungsvolle und machtorientierte Ziele}, sowie \emph{die Ver"anderung von sozialem Einfluss} im Gegensatz zu ``nur'' intrapsychischer Ver"anderung.

\subsubsection{Welche Bedeutung hat der Einfluss des Kontextes im Hinblick auf das Empowerment-Prozessmodell?}
Selbstverst"andlich eine sehr gro"se. Empowerment findet in einem Kontext statt, in dem Macht ungleich verteilt ist. Es ist au'serdem damit zu rechnen, dass bestimmte Gruppen an einer Aufrechterhaltung dieses Status Quo interessiert sind. Dementsprechend beeinflusst der Kontext den Erfolg von Empowerment-Bem"uhungen.

\subsubsection{\emph{Empowerment is participation} beschreibt Empowerment als weiterentwickeltes Konzept im Vergleich zu \emph{Empowerment is mastery}. Inwiefern?}
\emph{Empowerment is mastery} geht zur"uck auf Rappaport und definiert Empowerment als einen Mechanismus, durch den Menschen, Organisationen und Communities \emph{mastery} "uber ihre eigenen Angelegenheiten erlangen. Dies wurde kritisiert, weil es das Wohlergehen der Community vernachl"assigt. Anders gesagt: es ist ein Konflikt-orientiertes Modell, welches das Bed"urfnis nach sozialer Integration nicht ber"ucksichtigt (Riger, 1993). 

\emph{Empowerment is participation} greift diese Kritik auf. Empowerment beinhaltet hier respektvolle und reflektierte Partizipation innerhalb einer Communiy,  mit dem Ziel, gleicherma"sen Zugang zu Macht und Kontrolle f"ur alle zu erlangen.

\subsubsection{Wie l"a"st sich die Kritik von Riger (1993) zu verschiedenen Empowerment-Konzepten zusammenfassen?}
Wie schon bei der letzten Frage klar geworden ist, steht Frau Riger zum einen f"ur eine gleichberechtigte Stellung von Konflikt- und Machtorientierten "Uberlegungen auf der einen Seite und sozialen und Zugeh"origkeitsorientierten "Uberlegungen auf der anderen Seite. Sie ist in Sorge, dass auch nach erfolgreichem Empowerment-Prozess die soziale Beziehung zwischen den Gruppen nicht leidet (z.B. muslimische Frauen sollen mehr Rechte bekommen und immer noch eine gute Beziehung zu muslimischen M"annern haben).

Au"serdem legt sie wert darauf, nicht nur Selbstwirksamkeit im Bezug auf Empowerment zu betrachten, sondern auch zu schauen, inwiefern sozialer Ver"anderung tats"achlich stattfindet.

\subsubsection{Was unterscheidet das Konzept des \emph{psychological empowerment} von den "ubrigen Konzepten?}
Das Konzept ist detailierter als die "ubrigen Konzepte. Zimmerman schl"agt drei Gruppen von ``beobachtbaren Variablen'' (auf Individuumsebene) vor (siehe Abb. \ref{fig:cattaneo1}).

\begin{figure}[hb!]
        \begin{center}
                \begin{tikzpicture}[mindmap,font=\large\scshape,grow cyclic,every node/.style={concept, execute at begin node=\hskip0pt},concept color=black!20,
                        level 1/.append style={level distance=4cm,sibling distance=1cm,sibling angle=120}]
                        \node{Psychological Empowerment}
                        child{node{intrapersonal}}
                        child{node{interactional}}
                        child{node{behavioural}};
                \end{tikzpicture}
        \end{center}
        \caption{Die 3 Komponenten von Psychological Empowerment nach Zimmerman (1995)}
        \label{fig:cattaneo1}
\end{figure}

\subsubsection{Warum ist es den Autorinnen wichtig, Empowerment als iterativen Prozess zu betrachten? Was unterscheidet diese Betrachtungsweise von den anderen Konzepten?}

Nach dem Model von \textcite{cattaneo_process_2010} besteht der Empowerment-Prozess darin, dass Ziele formuliert werden, denen Handlungen folgen, die dann wiederum auf ihren Einfluss im sozialen Gef"uge hin bewertet werden. je nach Ausgang dieser Bewertung werden dann wieder Ziele formuliert und der Prozess wiederholt sich.

Bisherige Konzepte haben sich einzelne Indikatoren aus diesem Kreislauf herausgepickt und diese als Empowerment bezeichnet. Dadurch entstehen zum einen viele Konzeptualisierungen eines Konzepts. Zum anderen verliert man so auch die eigentlichen Konzepte aus den Augen, n"amlich die Ziele und die soziale Ver"anderung. Im schlimmsten Fall kann das dazu f"uhren, dass gar nicht das Empowerment gef"ordert wird, was die Betroffenen sich w"unschen.
\subsubsection{Wie argumentieren die Autorinnen f"ur die Erforderlichkeit, individuelle und weitgefasste soziale Aspekte in das Empowerment-Prozessmodell aufzunehmen?}
Weil beides wichtig f"ur Empowerment ist:
\begin{itemize}
        \item \textbf{Interaktion.} Empowerment findet in einem sozialen Kontext statt.
        \item \textbf{Kontroll"uberzeugungen.} Das Vertrauen in die eigenen F"ahigkeiten ist nicht nur intrapsychishc, sondern h"angt von Erfahrungen ab.
        \item \textbf{die Breite des Power-Begriffs.} Es gibt \emph{Power to}, \emph{Power over} und \emph{Power from}.
\end{itemize}

\subsubsection{Warum ist es wichtig, gleichzeitig p"ersonlich bedeutsame und machtorientierte Ziele zu ber"ucksichtigen?}
Erst wenn beide Arten von Zielen ber"ucksichtigt werden, spricht man von Empowerment. Es hat also mit der Definition des Begriffs zu tun. \emph{Pers"onlich bedeutsam} m"ussen die Ziele sein, damit Handlungsmotivation entsteht (wird mit Self-Determination Theory erkl"art). \emph{Machtorientiert} m"ussen die Ziele sein, damit auch tats"achlich soziale Ver"anderung eintreten kann.

\subsubsection{Welche Rolle spielt \emph{self-efficacy} im Empowerment-Prozess?}
Self-efficacy (Selbstwirksamkeit) beschreibt Erwartungen bez"uglich der Wirksamkeit des eigenen Handelns. Im Empowerment-Prozess hat das vor allem mit dem \emph{Impact} zu tun: Wer nicht erwartet, dass seine Handlungen eine Wirkung zeigen, der wird vermutlich erst gar nicht handeln.

\subsubsection{Wie h"angen \emph{knowledge}, \emph{competence}, \emph{action} und \emph{impact} zusammen?}
Siehe Fig. \ref{fig:cattaneo2}.
\begin{figure}[hb!]
        \begin{center}
                \begin{tikzpicture}[every node/.style={shape=rectangle,draw,rounded corners,text centered,node distance=1.5cm,text width=2cm}]
                        \node at (-6,0) (know) {Knowledge};
                        \node [right=of know] (skill) {Competence};
                        \node [right=of skill] (act) {Action};
                        \node [right=of act] (imp) {Impact};
                        \draw [->] (know) -- (skill);
                        \draw [->] (skill) -- (act);
                        \draw [->] (act) -- (imp);
                \end{tikzpicture}
        \end{center}
        \caption{Zusammenh"ange von \emph{knowledge}, \emph{competence}, \emph{action} und \emph{impact} im Empowerment-Prozessmodell von \textcite{cattaneo_process_2010}.}
        \label{fig:cattaneo2}
\end{figure}

\subsubsection{Bei weiterer Forschung: Aus welchen Gr"unden k"onnen qualitative, aus welchen quantitative Ans"atze sinnvoll sein?}
\begin{itemize}
        \item \emph{Qualitative Zug"ange:} In Bereichen mit wenig Befunden
        \item \emph{Quantitative Zug"ange:} Z.B. f"ur l"angsschnittliche Untersuchungen
\end{itemize}

\subsubsection{Was ist gemeint mit der Aussage: ``Empowerment is not a commodity to be aquired''}
Man kann nicht einfach jemanden empowern. Empowerment ist ein langer und schwieriger Prozess, der viel Arbeit erfordert.

\subsubsection{Inwiefern sind Beratende im Sinne des Empwerment-Prozessmodells ``Co-learner''?}
W"ahrend der Ratsuchende bereits Experte in Bezufg auf seine Angelegenheiten ist, muss der Beratende dar"uber erstmal etwas lernen. Er muss die Community und die Menschen und deren Historie kennenlernen, die Probleme erfassen, etc.

\subsubsection{Auf welche Art kann das Empowerment-Prozessmodell zu mehr Gerechtigkeit beitragen?}
Es kann helfen, zu Unrecht benachteiligten Gruppen zu mehr Rechten zu verhelfen. Dabei darf sich die Ungerechtigkeit aber nicht ins Gegenteil verkehren, sodass nachher die ``Anderen'' die Benachteiligten sind.
