\subsection{Fragen und Antworten zum Text von \textcite{simon_when_2013}}
\subsubsection{Auf welche Formen der Mobilisierung bezieht sich die bisherige Forschung zum Zusammenhang von dualer Identifikation und politischer Mobilisierung von Migranten? Wie sieht der Zusammenhang aus?}
Bisherige Forschung konzentriert sich auf legitime politische Mobilisierung, die auch von Mitgliedern der aufnehmenden Gesellschaft akzpetiert wird. Migranten werden so mobilisiert, wenn beide Identit"aten gut zusammen passen.

\subsubsection{Welche m"oglichen Risiken sind mit einer \emph{dual identity} verbunden? Wie k"onnen sie sich auswirken}
Die zwei Identit"aten einer dualen Identit"at k"onnen inkompatibel sein. Das kann zu \emph{Anomie} f"uhren und zu einer politischen Radikalisierung.

\subsubsection{Was ist den zentrale Hypothese? Welches Design wurde gew"ahlt? Was war die zentrale abh"angige Variable und mit welcher Begr"undung wurde sie gew"ahlt?}
Siehe Tab. \ref{tab:simon}.
\begin{table}[h!]
        \centering
        \begin{tabular}{r p{10cm}}
                \hline
                Hypothese & Bei dualer Identit"at f"uhrt wahrgenommene Inkompatibilit"at der Identit"aten zu politischer Radikalisierung.\\
                Design & Paneluntersuchung mit zwei Messzeitpunkten\\
                abh"angige Variable & Sympathie mit radikalen politischen Aktionen\\
                Begr"undung & tats"achlicher politischer Radikalismus ist selten und seine Beobachtung dauert zu lange\\
                \hline
        \end{tabular}
        \caption{Details zur Untersuchung.}
        \label{tab:simon}
\end{table}

\subsubsection{Was ist das zentrale Ergebnis? Benennen sie den Pr"adiktor, den Moderator und die abh"angige Variable!}
Hypothese wurde best"atigt. Bei \emph{hoher wahrgenommener Inkompatibilit"at} (Moderator) besteht ein positiver Zusammenhang zwischen \emph{dualer Identit"at} und \emph{Sympathie f"ur politischen Radikalismus}. (Siehe auch Fig. \ref{fig:simon1}.)

\begin{figure}[h!]
        \begin{center}
                \begin{tikzpicture}
                        [konzept/.style={text width=3cm, text centered, minimum height=1.5cm, shape=ellipse, draw}]
                        \node [konzept] at (-5,0) (dual) {Duale Identit"at};
                        \node [konzept] at (+5,0) (symp) {Sympathie f"ur Radikalismus};
                        \node [konzept] at (0,+2) (inko) {Identit"ats\-in\-kom\-pa\-ti\-bi\-li\-t"at};
                        \draw [->] (dual) -- (symp);
                        \draw [->] (inko) -- (0,0);
                \end{tikzpicture}
        \end{center}
        \caption{Moderierte Regression von Identit"atsinkompatibilit"at auf Duale Identit"at.}
        \label{fig:simon1}
\end{figure}

\subsubsection{Beschreiben Sie die Abbildung der Simple-Slope-Analysis (Fig.1, p.5)}
Die Abbildung zeigt die Regressionsgeraden der Regression von politischem Radikalismus auf duale Identit"at in Abhn"angigkeit von wahrgenommener Inkompatibilit"at. Bei hoher Inkompatibilit"at ist die Steigung positiv, bei niedrigerer Inkompatibilit"at dagegen leicht negativ.

\subsubsection{Welches Fazit wird hinsichtlich der kausalen Richtung der Beziehung zwischen dualer Identifikation und Sympathie f"ur Radikalismus bei Personen mit hoher Identit"atsinkompatibilit"at gezogen? Welche Analysen st"utzen diese Schlussfolgerung?}
Die Beziehung soll eben genau so rum sein, wie in der Hypothese formuliert: duale Identit"at bei hoher Identit"atsinkompatibilit"at f"uhrt zu Radikalismus. Um das zu untermauern wurden die gleichen Regressionsanalysen nochmal mit anderen Kriterien durchgef"uhrt:
\begin{itemize}
        \item \textbf{Duale Identifikation (t2).} Sympathie f"ur Radikalismus (t1) hatte einen negativen Effekt.
        \item \textbf{wahrgenommene Inkompatibilit"at (t2).} Sympathie f"ur Radikalismus (t1) hatte einen positiven Effekt.
        \item \textbf{Deren Produktterm.} Sympathie f"ur Radikalismus (t1) hatte keinen Effekt.
        \item \textbf{Bei denen mit hoher Identit"atsinkompatibilit"at:} Kein Effekt von Sympathie f"ur Radikalismus (t1) auf duale Identifikation (t2).
\end{itemize}

\noindent Nochmal auf deutsch: Sympathie f"ur politischen Radikalismus (t2) \emph{steigt} mit dualer Identifikation (t1) (bei Identit"atsinkompatibilit"at), aber duale Identifikation (t2) \emph{sinkt} mit Sympahtie f"ur politischen Radikalismus (t1). 

\subsubsection{Welche Rolle spielt die religi"ose Identit"at?}
Es gibt keine Hinweise darauf, dass religi"ose Identit"at Sympathie f"ur radikales Verhalten f"ordert. Wenn "uberhaupt, dann wirkt sie ihr entgegen. (Dieser Zusammenhang galt in der Studie aber nur f"ur die t"urkischen und haupts"achlich muslimischen Migranten.)

\subsubsection{Die Autorinnen schlagen vor, Identit"atsinkompatibilit"at im Rahmen von Mehrebenen-Analysen zu untersuchen. Was bedeutet dieser Vorschlag? Wie k"onnten die unterschiedlichen Ebenen zusammenh"angen? Wie k"onnte eine solche Untersuchung aussehen?}

Der soziale Kontext k"onnte durchaus eine Rolle bei der individuellen Erfahrung von Identit"atsinkompatibilit"at spielen. Es k"onnte z.B. der Fall sein, dass in bestimmten Gegenden Deutschlands die durchschnittliche Inkompatibilit"at h"oher ist als in anderen. 
