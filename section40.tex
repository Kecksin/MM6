\section{Community-psychologische Beratung}
\subsection{Fragen und Antworten zu \textcite{warschburger_theoretischer_2009}}
\subsubsection{Was sind nach McLeod (2004) notwendige Qualit"aten von Beratenden?}
\begin{itemize}
        \item Interpersonelle F"ahigkeiten
        \item Pers"onliche "Uberzeugungen
        \item Konzeptionelle F"ahigkeiten
        \item Pers"onliche Integrit"at
        \item Beherrschung der Beratungstechniken
        \item F"ahigkeit, soziale Systeme zu verstehen und mit ihnen zu arbeiten
\end{itemize}

\subsubsection{Was bedeutet die Garantie der Vertraulichkeit und wo ist sie gesetzlich geregelt?}
Informationen, von denen der Betroffene nicht will, dass sie bekannt werden, oder die nur einem begrenzten Personenkreis bekannt sind, d"urfen vom Berater nicht weitergegeben werden. Die Garantie ist im Strafgesetzbuch ($\S 203$ StGB) geregelt. 

\subsubsection{Was zeigte sich in der Befragung von Sharpley et al. (2004) hinsichtlich der Hemmschwelle f"ur den Kontakt mit einer Beraterin im Vergleich zu anderen Berufsgruppen?}
Die Hemmschwelle, einen Berater aufzusuchen, ist niedriger als f"ur andere Berufsgruppen (Psychologen, Sozialarbeiter oder Psychater). 

\subsubsection{Wie k"onnen Psychotherapie und Beratung voneinander abgegrenzt werden? Wo liegen Probleme bei der Abgrenzung?}
Die wohl trennsch"arfste Abgrenzung liefert der rechtliche Rahmen: Psychotherapeut ist ein gesch"utzter Begriff, Berater nicht. Ansonsten gibt es eine Vielzahl von Kriterien zur Abgrenzung. Allerdings ist das Problem, dass Therapie und Beratung gro"se "Uberschneidungen aufweisen. Eine Gegen"uberstellung von Merkmalen ist in Tab. \ref{tab:beratung} dargestellt.
\begin{table}[h!]
        \centering
        \begin{tabular}{p{7cm} p{7cm}}
                \hline
                Beratung & Psychotherapie\\
                \hline
                nicht gesch"utzter Begriff & gesch"utzter Begriff\\
                bei normativen Fragen, sozialen Konflikten & nur bei St"orungen\\
                ressourcenorientiert, pr"aventiv & kurativ\\
                eher von kurzer Dauer & eher von l"angerer Dauer\\
                \hline
        \end{tabular}
        \caption{Gegen"uberstellung von Unterschieden zwischen Beratung und Therapie. Die Kriterien eigenen sich jedoch nicht immer gut zur Abgrenzung. Das trennsch"arfste Kriterium ist das erste.}
        \label{tab:beratung}
\end{table}

\subsubsection{Nennen sie f"ur die Beratungsformen Mediation, Mentoring und Training Beispiel aus community-psychologischen Handlungsfeldern!}
\emph{Mediation} ist die Vermittlung zwischen Konfliktparteien. Ein Beispiel aus der Community Psychologie ist die Vermittlung zwischen zugezogenen Migranten und langj"ahrigen Einwohnern eines Stadtteils.\\

\noindent \emph{Mentoring} bezeichnet eine freiwillige und ehrenamtliche Unterst"utzungst"atigkeit, meist bei Berufsanf"angern. Ein Beispiel aus der Community Psychologie k"onnte ein Hilfsangebot f"ur Jugendliche Berufseinsteiger aus sozial schwachen Gebieten sein.\\

\noindent Der Begriff des \emph{Trainings} stellt den "Ubungscharakter von Ma"snahmen heraus. Ein Beispiel aus der Community Psychologie ist ein Konfliktl"osetraining f"ur Fanbeauftragte des heimischen Fu"sballclubs.

\subsection{Fragen und Antworten zu \textcite{warschburger_beratungsprozess_2009}}
\subsubsection{Was sind nach Saunders (1993) die vier Stufen der Inanspruchnahme von Beratung, die durchlaufen werden? Benennen sie Prozesse, die auf den jeweiligen Stufen relevant sind!}
\begin{itemize}
        \item \textbf{Problemwahrnehmung und Selbsthilfeversuche.} Wahrnehmung einer Ist/Soll-Diskrepanz,die negativ \emph{valent} und \emph{relevant} ist (\emph{Leidensdruck}). Versuche, das Problem selbst, mit Hilfe von Freunden und Verwandten und durch Beschaffung von Material und Informationen zu l"osen.
        \item \textbf{Akzeptanz von Beratung als probates Mittel.} Zentral sind \emph{positive Ergebniserwartungen}. Andere Faktoren sind \emph{Erfahrungen mit Beratung}, \emph{Kosten-Nutzen-Bilanz} und \emph{Normen bez"uglich Beratung}.
        \item \textbf{Entscheidung, Beratung aufzusuchen.} Kann aus \emph{eigenem Antrieb} oder durch \emph{Druck von au"sen} (Freunde, Arbeitgeber) geschehen. H"aufig wird ein \emph{Ausl"oser} berichtet. 
        \item \textbf{Kontaktaufnahme.} Die meisten Emfehlungen kommen von professionell t"atigen Personen aus dem Gesundheitswesen, aber auch von Freunden. Au"serdem wichtig sind "ortliche Lage und finanzielle Aspekte.
\end{itemize}

\subsubsection{Welche Probleme gibt es bei der empirischen Erforschung von Inanspruchnahme?}
Vor allem \emph{selektive Stichproben} und damit einhergehend \emph{geringe Validit"at} der Befunde. Andere Gruppen, vor allem diejenigen, die das Problem noch nicht wahrgenommen haben, sind schwer zu erreichen.

\subsubsection{Welche Faktoren nehmen Einfluss auf die Inanspruchnahme von Beratung?}
Siehe dazu Fig. \ref{fig:warschburger1}.
\begin{figure}[h!]
        \begin{center}
                \begin{tikzpicture}[node distance=1cm, faktor/.style={shape=rectangle,draw,text centered,rounded corners,minimum size=1.5cm, text width=4cm},
                        positiv/.style={shape=rectangle,draw,text centered,rounded corners,fill=green!20,minimum size=0.7cm,text width=4cm,font=\footnotesize,node distance=0.5cm},
                        negativ/.style={shape=rectangle,draw,text centered,rounded corners,fill=red!20,minimum size=0.7cm,text width=4cm,font=\footnotesize,node distance=0.5cm}]
                        %die 3 Faktoren
                        \node [faktor]at (0,0) (pers) {Personenmerkmale};
                        \node [faktor,right=of pers] (sozio) {Soziokulturelle Merkmale};
                        \node [faktor,right=of sozio] (orga) {Organisatorische Merkmale};
                        %positive Beispiele
                        \node [positiv, above= of pers] (weib) {weiblich};
                        \node [positiv, above= of weib] (bildung) {hohes Bildungsniveau};
                        \node [positiv, above= of sozio] (normen) {Normen};
                        \node [positiv, above= of normen] (normen) {soziale Unterst"utzung};
                        \node [positiv, above= of orga] (vertrau) {Vertraulichkeit};
                        \node [positiv, above= of vertrau] (gratis) {kostenloser Service};
                        %negative Beispiele
                        \node [negativ, below= of pers] (erwart) {negative Erwartungen};
                        \node [negativ, below= of erwart] (minor) {Minorit"atszugeh"origkeit};
                        \node [negativ, below= of sozio] (unterst) {soziale Unterst"utzung};
                        \node [negativ, below= of unterst] (stigma) {erwartete Stigmatisierung};
                        \node [negativ, below= of orga] (kenntnis) {fehlende Kenntnis "uber Angebote};
                        \node [negativ, below= of kenntnis] (ort) {bekannter Ort};
                \end{tikzpicture}
        \end{center}
        \caption{Einflussfaktoren auf Inanspruchnahme von Beratung.}
        \label{fig:warschburger1}
\end{figure}

\subsubsection{Welche Problematik kann sich bei pr"aventiven Angeboten ergeben?}
Ein Faktor bei der Inanspruchnahme von Beratung ist die Schwere der Problematik. Wenn das Problem nicht so schlimm ist, dann wird eher keine Beratung aufgesucht. Pr"aventive M"a"snahmen finden naturgem"a"s statt, wenn noch \emph{gar keine} Problematik vorliegt. Dementsprechend ist es schwer, Personen f"ur solche Ma"snahmen zu mobiliseren. 

\subsubsection{Welche vier Phasen des Beratungsprozesses werden unterschieden?}
\begin{itemize}
        \item Problemdefinition (Ist-Zustand)
        \item Zieldefinition (Soll-Zustand)
        \item Intervenieren (Schritte zur L"osung)
        \item Evaluation
\end{itemize}

\subsubsection{Was muss bei der Problemdefinition beachtet werden? Wie w"urde dieser Schritt im Rahmen einer Community aussehen und wie w"urde er gestaltet werden?}
Die ``objektive'' Sicht des Beraters und die des Klienten m"ussen nicht "ubereinstimmen. Z.B. auf Grund von selbstwertdienlichen Kausalattributionen oder aus Angst k"onnen Dinge bewusst oder unbewusst verschwiegen werden. 

\subsubsection{Was wird unter ressourcenorientierter Diagnostik verstanden?}
Gemeint ist das ausfindig machen von Ressourcen wie sozialer Unterst"utzung. Es geht darum, das Selbsthilfepotential des Klienten zu f"ordern.

\subsubsection{Nennen sie jeweils ein Beispiel f"ur Einflussfaktoren auf Beratung aus den vier Systemen des Modells von Bronfenbrenner!}
\begin{itemize}
        \item \textbf{Makrosystem} ("Ubergreifende kulturelle und gesellschaftliche Faktoren) 
                \begin{itemize}
                        \item Normen
                \end{itemize}
        \item \textbf{Exosystem} (Lebensbereiche, an denen die Person selbser nicht beteiligt ist, die sie aber indirekt betreffen)
                \begin{itemize}
                        \item Medien
                \end{itemize}
        \item \textbf{Mesosystem} (Wechselbeziehungen zwischen den Lebensbereichen einer Person)
                \begin{itemize}
                        \item Freunde
                \end{itemize}
        \item \textbf{Mikrosystem} (Lebensbereiche einer Person)
                \begin{itemize}
                        \item Arbeitskollegen
                \end{itemize}
\end{itemize}

\subsubsection{Welche Rolle spielt Evaluation im Beratungsgeschehen? Welche Arten von Verfahren k"onnen unterschieden werden? Wie sollte Evaluation von Beratung m"oglichst erfolgen?}
Eine wichtige. Man unterscheidet \emph{allgemeine} und \emph{st"orungsspezifische} Verfahren der Evaluation. Evaluation sollte nicht nur in Bezug auf das Ergebnis, sondern bereits \emph{prozessbegleitend} durchgef"uhrt werden.

\subsubsection{Was kann hinsichtlich der Wirksamkeitsforschung von Beratung festgestellt werden?}
Steht noch am Anfang. Beratung ist kein eigenst"andiger Beruf weshalb Beratungsans"atze gegen"uber klinischen Ans"atzen vernachl"assigt werden. F"ur Ans"atze, die sich in der Praxis bew"ahrt haben, liegen deshalb oft keine Studien vor.

\subsubsection{Welche Barrieren gibt es bei der Wirksamkeitsforschung von Beratung?}
\begin{itemize}
        \item keine einheitliche Definition von Beratung
        \item keine einheitliche Definition von Effektebenen (Symptombesserung, pr"aventive Ma"snahmen)
        \item mangelnde Abgrenzung zwischen Beratung und Psychotherapie
        \item Berater kommen aus unterschiedlichen Diszpiplinen
        \item zu viele Erhebungsinstrumente, keine Vergleichbarkeit
        \item unterschiedliche Settings mit unterschiedlicher Dauer
\end{itemize}

\subsubsection{Welcher Ansatz wies im Rahmen von schulbasierter Beratung die h"ochsten Effektst"arken auf?}
Kognitiv-behavioraler Ansatz, gefolgt von Entspannung und Fertigkeitsbasierten Ans"atzen.

\subsubsection{Inwiefern werden sogenannte Efficacy-Studien dem adapativen Charakter von Beratung nicht gerecht?}
In der Praxis wird eine Modifikation der Behandlungsmethode vorgenommen, wenn der Berater merkt, dass es nicht so l"auft. In Efficacy-Studien wird das ausgeschlossen.

\subsubsection{Welche Rolle spielt die Berater-Klient-Beziehung bei der Betrachtung der Wirkfaktoren?}
Die Berater-Kient-Beziehung ist die vielleicht wichtigste Variable in der Beratung. Sie ist der st"arkste Pr"adiktor f"ur den Behandlungserfolg. Sie geh"ort zu den allgemeinen Wirkfaktoren, weil sie unabh"angig von der Beratungsform einen gro"sen Einfluss hat.  
