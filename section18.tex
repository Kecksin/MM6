\subsection{Fragen und Antworten zum Text von \textcite{mayer_altersdiskriminierung_2009}}
\subsubsection{Was ist Ageism nach Robert Butler?}
Der Begriff umfasst drei Facetten:
\begin{itemize}
        \item Vorurteile gg"u. "alteren Menschen, dem Alter und dem Alternsprozess
        \item soziale Diskrimnierungen "alterer Menschen
        \item institutionelle und politische Praktiken, die Stereotype best"atigen und aufrechterhalten
\end{itemize}

\subsubsection{Wie definieren \textcite{mayer_altersdiskriminierung_2009} Altersdiskriminierung? In welchen F"allen sprechen sie nicht von Altersdiskriminierung, obwohl eine Beeintr"achtigung vorliegt?}
Zwei Punkte sind wichtig f"ur das Vorliegen von Altersdiskriminierung. Die Beeintr"achtigung\ldots
\begin{itemize}
        \item legitimer Anspr"uche
        \item allein aufgrund des Alters
\end{itemize}

Wenn die Diskriminerung z.B. aufgrund von alterskorellierten Merkmalen (Gesundheitszustand, fehlende technische Expertise) erfolgt, oder auch keine legitimen Anspr"uche verletzt (Vermeidung von Gespr"achen, Nicht-Aufnahme in ein junges Team), dann ist das zwar unerw"unscht, aber keine Altersdiskrimierung.

\subsubsection{In welchen Lebensbereichen findet Altersdiskriminierung statt?}
\begin{itemize}
        \item Arbeit und Beruf
                \begin{itemize}
                        \item "Altere nehmen seltener an Weiterbildungsma"snahmen teil
                        \item geringere Chance auf Einstellung
                \end{itemize}
        \item Gesundheitswesen
                \begin{itemize}
                        \item Qualitaet der medizinischen Versorgung
                        \item fehlende Wertsch"atzung in der Interaktion mit dem Arzt
                \end{itemize}
        \item Pflege
                \begin{itemize}
                        \item Babysprache
                        \item Fixierung
                \end{itemize}
        \item Medien
                \begin{itemize}
                        \item es werden nur vitale alte Menschen gezeigt
                        \item oder nur "uber Risiken gesprochen
                \end{itemize}
        \item Rechtswesen
                \begin{itemize}
                        \item Glaubw"urdigkeit bei Zeugenaussagen
                \end{itemize}
\end{itemize}

\subsubsection{Welche individuellen Ans"atze zur Erkl"arung von Altersdiskriminierung werden vorgestellt?}
\begin{itemize}
        \item \textbf{Altersstereotype.} Negative und auch positive Einstellungen gg"u. alten Menschen und dem Altern.
        \item \textbf{"Angste vor Altern und Tod.} Nach der \emph{Terror Management Theory} erinnert die Begegnung mit dem Alter an die eigene Zerbrechlichkeit und kann deshalb als bedrohlich empfunden werden.
        \item \textbf{Alterspezifische Motivationen.} Nach der Theorie der sozio-emotionalen selektivit"at suchen j"ungere Menschen ihre Interaktionspartner eher nach einem m"oglichen Informationsgewinn aus. "Altere Menschen suchen eher emptionale Bed"urfnisbefriedigung.
\end{itemize}

\subsubsection{Welche strukturellen Erkl"arungsans"atze werden vorgestellt?}
\begin{itemize}
        \item \textbf{Intergenerationelle Konflikte.} Es geht um Gruppenkonflikte um begrenzte Ressourcen.
        \item \textbf{Strukturelle und institutionelle Rahmenbedingungen.} Durch den demographischen Wandel steht nicht mehr genug Pflegepersonal zu Verf"ugung.
\end{itemize}
\subsubsection{Welche Vor- und Nachteile hat Diskriminierung f"ur die Dskrimierten und f"ur die Diskriminierenden?}
Nachteile f"ur die Diskriminierten:
\begin{itemize}
        \item finanzielle Verluste
        \item gesundheitliche Sch"aden
        \item R"uckzug aus sozialem Leben
        \item Kompletenzverluste im Sinne selbsterf"ullender Prophezeiungen
        \item Folgen der Aktivierung eines negativen Altersstereotyps (Selbstbild, Leistungsf"ahigkeit, Lebenserwartung)
\end{itemize}

\noindent Bew"altigungsprozesse zur Abpufferung:
\begin{itemize}
        \item Menschen kennen die Schattenseiten des "Alterwerdens, sch"atzen aber die Wahrscheinlichkeit gering ein, selbst betroffen sein zu werden
        \item soziale Abw"artsvergleiche
        \item Distanzierung von der eigenen Altersgruppe
\end{itemize}

\noindent Vorteile f"ur die Diskriminierenden:
\begin{itemize}
        \item Stabilisierung und F"orderung des Selbstwergef"uhls
        \item soziale Zugeh"origkeit und positive Eigengruppenidentit"at
        \item Legitimation sozialer Ungleichheit
        \item Schutz vor negativen altersbezogenen Emotionen
\end{itemize}

\noindent Nachteile f"ur die Diskriminierenden:
\begin{itemize}
        \item Verlust der Potentiale der "alterer Menschen
        \item selbstsch"adigendes Verhalten (irgendwann geh"ort jeder zur Gruppe der "alteren Menschen)
        \item volkswirtschaftliche Kosten
\end{itemize}

\subsubsection{Welche Ma"snahmen zum Abbau und zur Pr"avention von Altersdiskriminierung lassen sich ableiten?}
Individuumszentrierte Ma"snahmen:
\begin{itemize}
        \item F"orderung intergenerationaler Kontakte
        \item Wissensvermittlung
        \item Rollenspiele und Alterssimulationen
        \item F"orderung der Perspektiven"ubernahme
        \item Aufbau von Kompetenzen im Umgang mit "alteren Menschen
\end{itemize}

\noindent Strukturelle Ma"snahmen:
\begin{itemize}
        \item Gesetze (Antidiskrimierungsgesetze, Flexibilisierung von Arbeitszeiten)
        \item Verordnungen (Bauvorschriften, "OPNV)
        \item "Offentlichkeitsarbeit f"ur ein differenziertes und realistisches Altersbild
\end{itemize}
